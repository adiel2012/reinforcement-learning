\documentclass[11pt,twoside,openright]{book}

% Essential packages for professional typography
\usepackage[utf8]{inputenc}
\usepackage[T1]{fontenc}
\usepackage{lmodern}
\usepackage[english]{babel}
\usepackage{geometry}
\usepackage{fancyhdr}
\usepackage{graphicx}
\usepackage{amsmath,amsfonts,amssymb,amsthm}
\usepackage{mathtools}
\usepackage{algorithm}
\usepackage{algorithmic}
\usepackage{xcolor}
\usepackage{tcolorbox}
\usepackage{hyperref}
\usepackage{cleveref}
\usepackage{booktabs}
\usepackage{float}
\usepackage{enumitem}
\usepackage{microtype}  % Better typography
\usepackage{setspace}   % Line spacing control
\usepackage{parskip}    % Better paragraph spacing

% Page geometry - optimized for readability
\geometry{
    a4paper,
    left=3cm,
    right=2.5cm,
    top=3cm,
    bottom=3cm,
    bindingoffset=0.5cm,
    headheight=14pt
}

% Enhanced color scheme for better readability
\definecolor{theoremcolor}{RGB}{0,100,150}
\definecolor{examplecolor}{RGB}{150,100,0}
\definecolor{remarkcolor}{RGB}{100,150,0}
\definecolor{intuitioncolor}{RGB}{120,60,180}
\definecolor{keyideacolor}{RGB}{200,50,50}
\definecolor{applicationcolor}{RGB}{50,150,80}
\definecolor{warningcolor}{RGB}{255,140,0}
\definecolor{notecolor}{RGB}{70,130,180}

% Theorem environments with professional styling
\newtheorem{theorem}{Theorem}[chapter]
\newtheorem{lemma}[theorem]{Lemma}
\newtheorem{corollary}[theorem]{Corollary}
\newtheorem{definition}[theorem]{Definition}
\newtheorem{proposition}[theorem]{Proposition}
\newtheorem{example}[theorem]{Example}
\newtheorem{assumption}[theorem]{Assumption}
\newtheorem{remark}[theorem]{Remark}

% Enhanced box environments
\tcbuselibrary{theorems,skins,breakable}

\newtcolorbox{keyideabox}[1][]{
    enhanced,
    colback=keyideacolor!5,
    colframe=keyideacolor,
    title=#1,
    fonttitle=\bfseries,
    rounded corners,
    drop shadow,
    breakable,
    left=5pt,
    right=5pt,
    top=5pt,
    bottom=5pt
}

\newtcolorbox{intuitionbox}[1][]{
    enhanced,
    colback=intuitioncolor!5,
    colframe=intuitioncolor,
    title=#1,
    fonttitle=\bfseries,
    rounded corners,
    breakable,
    left=5pt,
    right=5pt,
    top=5pt,
    bottom=5pt
}

\newtcolorbox{examplebox}[1][]{
    enhanced,
    colback=examplecolor!5,
    colframe=examplecolor,
    title=#1,
    fonttitle=\bfseries,
    rounded corners,
    drop shadow,
    breakable,
    left=5pt,
    right=5pt,
    top=5pt,
    bottom=5pt
}

\newtcolorbox{remarkbox}[1][]{
    enhanced,
    colback=remarkcolor!5,
    colframe=remarkcolor,
    title=#1,
    fonttitle=\bfseries,
    rounded corners,
    breakable,
    left=5pt,
    right=5pt,
    top=5pt,
    bottom=5pt
}

\newtcolorbox{notebox}[1][]{
    enhanced,
    colback=notecolor!5,
    colframe=notecolor,
    title=#1,
    fonttitle=\bfseries,
    rounded corners,
    breakable,
    left=5pt,
    right=5pt,
    top=5pt,
    bottom=5pt
}

\newtcolorbox{applicationbox}[1][]{
    enhanced,
    colback=applicationcolor!5,
    colframe=applicationcolor,
    title=#1,
    fonttitle=\bfseries,
    rounded corners,
    breakable,
    left=5pt,
    right=5pt,
    top=5pt,
    bottom=5pt
}

\newtcolorbox{warningbox}[1][]{
    enhanced,
    colback=warningcolor!5,
    colframe=warningcolor,
    title=#1,
    fonttitle=\bfseries,
    rounded corners,
    breakable,
    left=5pt,
    right=5pt,
    top=5pt,
    bottom=5pt
}

% Custom commands for RL notation
\newcommand{\state}{\mathcal{S}}
\newcommand{\action}{\mathcal{A}}
\newcommand{\reward}{\mathcal{R}}
\newcommand{\policy}{\pi}
\newcommand{\valuefunction}{V}
\newcommand{\qvalue}{Q}
\newcommand{\transition}{P}
\newcommand{\discount}{\gamma}
\newcommand{\expect}{\mathbb{E}}
\newcommand{\real}{\mathbb{R}}
\newcommand{\naturals}{\mathbb{N}}
\newcommand{\integers}{\mathbb{Z}}
\newcommand{\prob}{\mathbb{P}}
\newcommand{\indicator}{\mathbf{1}}
\newcommand{\argmax}{\operatorname{argmax}}
\newcommand{\argmin}{\operatorname{argmin}}
\newcommand{\minimize}{\operatorname{minimize}}
\newcommand{\maximize}{\operatorname{maximize}}
\newcommand{\trace}{\operatorname{trace}}
\newcommand{\rank}{\operatorname{rank}}
\newcommand{\diag}{\operatorname{diag}}
\newcommand{\vectorize}{\operatorname{vec}}

% Header and footer styling
\pagestyle{fancy}
\fancyhf{}
\fancyhead[LE]{\leftmark}
\fancyhead[RO]{\rightmark}
\fancyfoot[C]{\thepage}
\renewcommand{\headrulewidth}{0.4pt}

% Hyperref setup for better PDF output
\hypersetup{
    colorlinks=true,
    linkcolor=blue,
    filecolor=magenta,
    urlcolor=cyan,
    citecolor=green,
    pdftitle={Reinforcement Learning for Engineer-Mathematicians},
    pdfauthor={Enhanced Edition},
    pdfsubject={Reinforcement Learning},
    pdfkeywords={reinforcement learning, engineering, mathematics, control theory, machine learning}
}

% Better line spacing
\onehalfspacing

\begin{document}

% Front matter
\frontmatter

% Title page
\begin{titlepage}
    \centering
    \vspace*{2cm}
    
    {\Huge\bfseries Reinforcement Learning\\for Engineer-Mathematicians\par}
    \vspace{1.5cm}
    {\Large A Comprehensive Guide to Theory and Applications\par}
    \vspace{2cm}
    
    \begin{tcolorbox}[colback=blue!5,colframe=blue!40!black,title=About This Enhanced Edition]
    This comprehensive textbook bridges theory and practice in reinforcement learning:
    \begin{itemize}
        \item \textbf{18 Complete Chapters}: From mathematical foundations to research frontiers
        \item \textbf{Mathematical Rigor}: Formal theorems, proofs, and convergence analysis
        \item \textbf{Practical Focus}: Engineering applications and implementation guidance  
        \item \textbf{Interactive Learning}: 5 Jupyter notebooks with Google Colab support
        \item \textbf{Modern Coverage}: Classical methods through deep reinforcement learning
    \end{itemize}
    \end{tcolorbox}
    
    \vspace{2cm}
    {\large Enhanced Edition 2024\par}
    \vfill
    
    {\large Publisher Information\par}
\end{titlepage}

% Copyright page
\newpage
\thispagestyle{empty}
\vspace*{\fill}
\begin{center}
Copyright \copyright\ 2024 Enhanced Edition\\
All rights reserved under Creative Commons Attribution-ShareAlike 4.0\\[1em]
\textbf{Open Educational Resource}\\
This work is licensed under CC BY-SA 4.0\\[1em]
\textbf{Repository}: \url{https://github.com/adiel2012/reinforcement-learning}
\end{center}
\vspace*{\fill}

% Dedication
\newpage
\thispagestyle{empty}
\vspace*{\fill}
\begin{center}
\textit{To all engineers and mathematicians who seek to bridge\\
the gap between theory and practice in the age of intelligent systems}
\end{center}
\vspace*{\fill}

% Preface
\chapter*{Preface}
\addcontentsline{toc}{chapter}{Preface}

This enhanced edition represents a comprehensive treatment of reinforcement learning designed specifically for engineer-mathematicians who require both theoretical rigor and practical implementation guidance. The field of reinforcement learning has evolved rapidly, with breakthroughs in deep learning enabling applications previously thought impossible.

\section*{What Makes This Edition Special}

This textbook bridges the gap between mathematical theory and engineering practice through:

\begin{itemize}
\item \textbf{Complete Coverage}: 18 chapters covering foundations through research frontiers
\item \textbf{Mathematical Rigor}: Formal definitions, theorems, proofs, and convergence analysis
\item \textbf{Engineering Focus}: Practical implementations and real-world applications
\item \textbf{Interactive Learning}: Jupyter notebooks with Google Colab support
\item \textbf{Modern Approach}: Classical methods through state-of-the-art deep RL
\end{itemize}

\section*{How to Use This Book}

The book is designed for flexible learning:

\begin{description}
\item[Theory-First Learners] Begin with the mathematical foundations in Chapters 1-5, then explore advanced topics
\item[Hands-On Learners] Start with the interactive notebooks for Chapters 1-5, then delve into theory
\item[Practitioners] Focus on implementation chapters (6-8, 14-17) with theoretical background as needed
\item[Researchers] Use as a comprehensive reference for both classical and modern methods
\end{description}

\section*{Acknowledgments}

Special thanks to the reinforcement learning community for their open sharing of ideas, the developers of open-source tools that make interactive learning possible, and the countless researchers who have advanced this fascinating field.

% Table of contents
\tableofcontents

% Main matter
\mainmatter

% Part I: Mathematical Foundations
\part{Mathematical Foundations}

This part establishes the mathematical foundations necessary for understanding reinforcement learning from both theoretical and engineering perspectives. We begin with essential mathematical prerequisites, develop the formal framework of Markov Decision Processes, and conclude with classical dynamic programming methods.

The treatment emphasizes mathematical rigor while maintaining practical relevance for engineering applications. Each concept is developed with careful attention to assumptions, proofs, and connections to control theory and optimization.

\chapter{Introduction and Mathematical Prerequisites}
\label{ch:introduction}

\begin{keyideabox}[Chapter Overview]
This chapter introduces the fundamental mathematical tools needed for reinforcement learning and provides intuitive motivation for why RL represents a paradigm shift from classical control theory. We'll cover probability theory, linear algebra, optimization, and stochastic processes with practical examples.
\end{keyideabox}

\section{Motivation: From Control Theory to Learning Systems}

\begin{intuitionbox}[Why Reinforcement Learning?]
Imagine teaching a child to ride a bicycle. You don't give them the equations of motion or tell them exactly how to balance. Instead, they learn through trial and error, gradually improving their balance and control. This is the essence of reinforcement learning.
\end{intuitionbox}

Reinforcement learning represents a fundamental paradigm shift from classical control theory and optimization. While traditional engineering approaches rely on explicit models and well-defined objectives, reinforcement learning enables systems to learn optimal behavior through interaction with their environment.

\begin{examplebox}[Traditional Control vs. Reinforcement Learning]
Consider a classic engineering problem: designing a controller for an inverted pendulum.

\textbf{Traditional Control Approach:}
\begin{enumerate}
    \item Deriving the system dynamics using Lagrangian mechanics
    \item Linearizing around the equilibrium point  
    \item Designing a feedback controller using pole placement or LQR
    \item Implementing the controller with known parameters
\end{enumerate}

\textbf{Reinforcement Learning Approach:}
\begin{enumerate}
    \item Define states (angle, angular velocity) and actions (applied force)
    \item Specify a reward function (positive for upright, negative for falling)
    \item Allow the agent to explore and learn through trial and error
    \item Converge to an optimal policy without explicit knowledge of dynamics
\end{enumerate}
\end{examplebox}

This fundamental difference opens up possibilities for systems where:
\begin{itemize}
    \item Dynamics are unknown or too complex to model accurately
    \item Environment conditions change over time
    \item Multiple conflicting objectives must be balanced
    \item System parameters vary or degrade over time
\end{itemize}

\begin{examplebox}[Industrial Example: Power Grid Management]
Modern power grids face unprecedented challenges with renewable energy integration, electric vehicle charging, and dynamic pricing. Traditional grid control relies on pre-computed lookup tables and heuristic rules. Reinforcement learning enables real-time optimization that adapts to:
\begin{itemize}
    \item Variable renewable generation
    \item Changing demand patterns
    \item Equipment failures and network topology changes
    \item Market price fluctuations
\end{itemize}
\end{examplebox}

\section{Mathematical Notation and Conventions}

\begin{notebox}[Notation Guide]
Throughout this book, we adopt consistent mathematical notation that aligns with both control theory and machine learning conventions. Don't worry if some symbols are unfamiliar now—we'll introduce them gradually with intuitive explanations.
\end{notebox}

Throughout this book, we adopt consistent mathematical notation that aligns with both control theory and machine learning conventions.

\subsection{Sets and Spaces}

\begin{intuitionbox}[Understanding Spaces]
Think of a "space" as the collection of all possible values a variable can take. For example, if we're controlling a robot arm, the state space might include all possible joint angles and velocities.
\end{intuitionbox}

\begin{align}
\state &= \text{State space (all possible states)} \\
\action &= \text{Action space (all possible actions)} \\
\reward &= \text{Reward space (all possible rewards)} \\
\real^n &= \text{$n$-dimensional real vector space} \\
\real^{m \times n} &= \text{Space of $m \times n$ real matrices}
\end{align}

\subsection{Functions and Operators}
\begin{align}
\policy: \state \to \action &\quad \text{(Deterministic policy)} \\
\policy: \state \to \Delta(\action) &\quad \text{(Stochastic policy)} \\
\valuefunction^\policy: \state \to \real &\quad \text{(Value function)} \\
\qvalue^\policy: \state \times \action \to \real &\quad \text{(Action-value function)} \\
T: \real^\state \to \real^\state &\quad \text{(Bellman operator)}
\end{align}

where $\Delta(\action)$ denotes the space of probability distributions over $\action$.

\subsection{Probability and Expectation}
\begin{align}
\prob(s'|s,a) &= \text{Transition probability} \\
\expect_\policy[\cdot] &= \text{Expectation under policy $\policy$} \\
\expect_{s \sim \mu}[\cdot] &= \text{Expectation over distribution $\mu$}
\end{align}

\section{Probability Theory Refresher}

Reinforcement learning is fundamentally about making decisions under uncertainty. A solid understanding of probability theory is essential for analyzing convergence properties, sample complexity, and algorithm performance.

\subsection{Probability Spaces and Random Variables}

\begin{definition}[Probability Space]
A probability space is a triple $(\Omega, \mathcal{F}, \prob)$ where:
\begin{itemize}
    \item $\Omega$ is the sample space (set of all possible outcomes)
    \item $\mathcal{F}$ is a $\sigma$-algebra on $\Omega$ (collection of measurable events)
    \item $\prob: \mathcal{F} \to [0,1]$ is a probability measure satisfying:
    \begin{enumerate}
        \item $\prob(\Omega) = 1$
        \item For disjoint events $A_1, A_2, \ldots$: $\prob(\bigcup_{i=1}^\infty A_i) = \sum_{i=1}^\infty \prob(A_i)$
    \end{enumerate}
\end{itemize}
\end{definition}

\begin{definition}[Random Variable]
A random variable $X$ is a measurable function $X: \Omega \to \real$ such that for every Borel set $B \subseteq \real$, the preimage $X^{-1}(B) \in \mathcal{F}$.
\end{definition}

\subsection{Conditional Expectation and Martingales}

Conditional expectation plays a crucial role in reinforcement learning, particularly in the analysis of temporal difference methods and policy gradient algorithms.

\begin{definition}[Conditional Expectation]
Given random variables $X$ and $Y$, the conditional expectation $\expect[X|Y]$ is the unique (almost surely) random variable that is:
\begin{enumerate}
    \item Measurable with respect to $\sigma(Y)$
    \item Satisfies $\expect[\expect[X|Y] \cdot \mathbf{1}_A] = \expect[X \cdot \mathbf{1}_A]$ for all $A \in \sigma(Y)$
\end{enumerate}
\end{definition}

\begin{theorem}[Law of Total Expectation]
For random variables $X$ and $Y$:
\begin{equation}
\expect[X] = \expect[\expect[X|Y]]
\end{equation}
\end{theorem}

\begin{definition}[Martingale]
A sequence of random variables $\{X_t\}_{t=0}^\infty$ is a martingale with respect to filtration $\{\mathcal{F}_t\}_{t=0}^\infty$ if:
\begin{enumerate}
    \item $X_t$ is $\mathcal{F}_t$-measurable for all $t$
    \item $\expect[|X_t|] < \infty$ for all $t$
    \item $\expect[X_{t+1}|\mathcal{F}_t] = X_t$ almost surely
\end{enumerate}
\end{definition}

Martingales are fundamental for proving convergence of stochastic algorithms in reinforcement learning.

\subsection{Concentration Inequalities}

Concentration inequalities provide bounds on the probability that random variables deviate from their expected values. These are essential for finite-sample analysis of RL algorithms.

\begin{theorem}[Hoeffding's Inequality]
Let $X_1, \ldots, X_n$ be independent random variables with $X_i \in [a_i, b_i]$ almost surely. Then for any $t > 0$:
\begin{equation}
\prob\left(\left|\frac{1}{n}\sum_{i=1}^n X_i - \frac{1}{n}\sum_{i=1}^n \expect[X_i]\right| \geq t\right) \leq 2\exp\left(-\frac{2n^2t^2}{\sum_{i=1}^n(b_i-a_i)^2}\right)
\end{equation}
\end{theorem}

\begin{theorem}[Azuma's Inequality]
Let $\{X_t\}_{t=0}^\infty$ be a martingale with respect to $\{\mathcal{F}_t\}_{t=0}^\infty$ such that $|X_{t+1} - X_t| \leq c_t$ almost surely. Then:
\begin{equation}
\prob(|X_n - X_0| \geq t) \leq 2\exp\left(-\frac{t^2}{2\sum_{i=0}^{n-1}c_i^2}\right)
\end{equation}
\end{theorem}

\section{Linear Algebra Essentials}

Linear algebra provides the foundation for function approximation, policy parameterization, and many algorithmic techniques in reinforcement learning.

\subsection{Vector Spaces and Norms}

\begin{definition}[Vector Space]
A vector space $V$ over field $\mathbb{F}$ (typically $\real$ or $\mathbb{C}$) is a set equipped with vector addition and scalar multiplication satisfying:
\begin{enumerate}
    \item Commutativity: $u + v = v + u$
    \item Associativity: $(u + v) + w = u + (v + w)$
    \item Identity: $\exists 0 \in V$ such that $v + 0 = v$
    \item Inverse: $\forall v \in V, \exists -v$ such that $v + (-v) = 0$
    \item Scalar associativity: $a(bv) = (ab)v$
    \item Scalar identity: $1v = v$
    \item Distributivity: $a(u + v) = au + av$ and $(a + b)v = av + bv$
\end{enumerate}
\end{definition}

\begin{definition}[Norm]
A norm on vector space $V$ is a function $\|\cdot\|: V \to \real_{\geq 0}$ satisfying:
\begin{enumerate}
    \item $\|v\| = 0$ if and only if $v = 0$
    \item $\|av\| = |a|\|v\|$ for scalar $a$
    \item $\|u + v\| \leq \|u\| + \|v\|$ (triangle inequality)
\end{enumerate}
\end{definition}

Common norms in $\real^n$:
\begin{align}
\|x\|_1 &= \sum_{i=1}^n |x_i| \quad \text{($\ell_1$ norm)} \\
\|x\|_2 &= \sqrt{\sum_{i=1}^n x_i^2} \quad \text{(Euclidean norm)} \\
\|x\|_\infty &= \max_{i=1,\ldots,n} |x_i| \quad \text{($\ell_\infty$ norm)}
\end{align}

\subsection{Inner Products and Orthogonality}

\begin{definition}[Inner Product]
An inner product on real vector space $V$ is a function $\langle \cdot, \cdot \rangle: V \times V \to \real$ satisfying:
\begin{enumerate}
    \item Symmetry: $\langle u, v \rangle = \langle v, u \rangle$
    \item Linearity: $\langle au + bv, w \rangle = a\langle u, w \rangle + b\langle v, w \rangle$
    \item Positive definiteness: $\langle v, v \rangle \geq 0$ with equality iff $v = 0$
\end{enumerate}
\end{definition}

The induced norm is $\|v\| = \sqrt{\langle v, v \rangle}$.

\begin{theorem}[Cauchy-Schwarz Inequality]
For vectors $u, v$ in an inner product space:
\begin{equation}
|\langle u, v \rangle| \leq \|u\| \|v\|
\end{equation}
with equality if and only if $u$ and $v$ are linearly dependent.
\end{theorem}

\subsection{Eigenvalues and Spectral Theory}

\begin{definition}[Eigenvalue and Eigenvector]
For matrix $A \in \real^{n \times n}$, scalar $\lambda$ is an eigenvalue with corresponding eigenvector $v \neq 0$ if:
\begin{equation}
Av = \lambda v
\end{equation}
\end{definition}

\begin{theorem}[Spectral Theorem for Symmetric Matrices]
Every real symmetric matrix $A$ has an orthonormal basis of eigenvectors with real eigenvalues. If $A = Q\Lambda Q^T$ where $Q$ is orthogonal and $\Lambda$ is diagonal, then:
\begin{equation}
A = \sum_{i=1}^n \lambda_i q_i q_i^T
\end{equation}
where $\lambda_i$ are eigenvalues and $q_i$ are corresponding orthonormal eigenvectors.
\end{theorem}

\section{Optimization Fundamentals}

Optimization theory underpins virtually all reinforcement learning algorithms, from value iteration to policy gradient methods.

\subsection{Convex Analysis}

\begin{definition}[Convex Set]
A set $C \subseteq \real^n$ is convex if for all $x, y \in C$ and $\lambda \in [0,1]$:
\begin{equation}
\lambda x + (1-\lambda) y \in C
\end{equation}
\end{definition}

\begin{definition}[Convex Function]
A function $f: \real^n \to \real$ is convex if its domain is convex and for all $x, y$ in the domain and $\lambda \in [0,1]$:
\begin{equation}
f(\lambda x + (1-\lambda) y) \leq \lambda f(x) + (1-\lambda) f(y)
\end{equation}
\end{definition}

\begin{theorem}[First-Order Characterization of Convexity]
For differentiable function $f$, the following are equivalent:
\begin{enumerate}
    \item $f$ is convex
    \item $f(y) \geq f(x) + \nabla f(x)^T(y - x)$ for all $x, y$
    \item $\nabla f$ is monotone: $(\nabla f(x) - \nabla f(y))^T(x - y) \geq 0$
\end{enumerate}
\end{theorem}

\subsection{Unconstrained Optimization}

\begin{theorem}[Necessary Conditions for Optimality]
If $x^*$ is a local minimum of differentiable function $f$, then:
\begin{equation}
\nabla f(x^*) = 0
\end{equation}
If $f$ is twice differentiable, then additionally:
\begin{equation}
\nabla^2 f(x^*) \succeq 0 \quad \text{(positive semidefinite)}
\end{equation}
\end{theorem}

\begin{theorem}[Sufficient Conditions for Optimality]
If $\nabla f(x^*) = 0$ and $\nabla^2 f(x^*) \succ 0$ (positive definite), then $x^*$ is a strict local minimum.
\end{theorem}

\subsection{Gradient Descent and Convergence Analysis}

The gradient descent algorithm iterates:
\begin{equation}
x_{k+1} = x_k - \alpha_k \nabla f(x_k)
\end{equation}

\begin{theorem}[Convergence of Gradient Descent]
For convex function $f$ with $L$-Lipschitz gradient and step size $\alpha \leq 1/L$:
\begin{equation}
f(x_k) - f(x^*) \leq \frac{\|x_0 - x^*\|^2}{2\alpha k}
\end{equation}
where $x^*$ is the optimal solution.
\end{theorem}

For strongly convex functions, the convergence rate improves to exponential.

\section{Stochastic Processes and Markov Chains}

Understanding stochastic processes is crucial for analyzing the temporal dynamics of reinforcement learning systems.

\subsection{Discrete-Time Stochastic Processes}

\begin{definition}[Stochastic Process]
A discrete-time stochastic process is a sequence of random variables $\{X_t\}_{t=0}^\infty$ where each $X_t$ takes values in some state space $\state$.
\end{definition}

\begin{definition}[Markov Property]
A stochastic process $\{X_t\}_{t=0}^\infty$ satisfies the Markov property if:
\begin{equation}
\prob(X_{t+1} = s' | X_t = s, X_{t-1} = s_{t-1}, \ldots, X_0 = s_0) = \prob(X_{t+1} = s' | X_t = s)
\end{equation}
for all states $s, s', s_0, \ldots, s_{t-1}$ and times $t \geq 0$.
\end{definition}

\subsection{Markov Chain Analysis}

For finite state space $\state = \{1, 2, \ldots, n\}$, a Markov chain is characterized by its transition matrix $P \in \real^{n \times n}$ where $P_{ij} = \prob(X_{t+1} = j | X_t = i)$.

\begin{definition}[Irreducibility and Aperiodicity]
A Markov chain is:
\begin{itemize}
    \item \textbf{Irreducible} if every state is reachable from every other state
    \item \textbf{Aperiodic} if $\gcd\{n \geq 1 : P_{ii}^{(n)} > 0\} = 1$ for some state $i$
\end{itemize}
\end{definition}

\begin{theorem}[Fundamental Theorem of Markov Chains]
For an irreducible, aperiodic, finite Markov chain:
\begin{enumerate}
    \item There exists a unique stationary distribution $\pi$ satisfying $\pi = \pi P$
    \item $\lim_{t \to \infty} P^t = \mathbf{1}\pi^T$ where $\mathbf{1}$ is the vector of ones
    \item For any initial distribution $\mu_0$: $\lim_{t \to \infty} \|\mu_t - \pi\|_{TV} = 0$
\end{enumerate}
\end{theorem}

\subsection{Mixing Times and Convergence Rates}

\begin{definition}[Total Variation Distance]
The total variation distance between distributions $\mu$ and $\nu$ on finite space $\state$ is:
\begin{equation}
\|\mu - \nu\|_{TV} = \frac{1}{2}\sum_{s \in \state} |\mu(s) - \nu(s)|
\end{equation}
\end{definition}

\begin{definition}[Mixing Time]
The mixing time of a Markov chain is:
\begin{equation}
t_{mix}(\epsilon) = \min\{t : \max_{i \in \state} \|P^t(i, \cdot) - \pi\|_{TV} \leq \epsilon\}
\end{equation}
\end{definition}

Understanding mixing times is essential for analyzing sample complexity in reinforcement learning algorithms that rely on sampling from stationary distributions.

\section{Chapter Summary}

This chapter established the mathematical foundations necessary for rigorous analysis of reinforcement learning algorithms. Key concepts include:

\begin{itemize}
    \item The paradigm shift from model-based control to learning-based optimization
    \item Probability theory tools: conditional expectation, martingales, concentration inequalities
    \item Linear algebra foundations: vector spaces, norms, spectral theory
    \item Convex optimization and gradient descent convergence analysis
    \item Markov chain theory and convergence to stationary distributions
\end{itemize}

These mathematical tools will be applied throughout the book to analyze algorithm convergence, sample complexity, and performance guarantees. The next chapter develops the formal framework of Markov Decision Processes, which provides the mathematical foundation for all subsequent reinforcement learning algorithms.
\chapter{Markov Decision Processes (MDPs)}
\label{ch:mdps}

\begin{keyideabox}[Chapter Overview]
This chapter introduces Markov Decision Processes (MDPs), the mathematical foundation of reinforcement learning. We'll build intuition through concrete examples before diving into the formal theory, then explore solution methods like value iteration and policy iteration.
\end{keyideabox}

\begin{intuitionbox}[What is an MDP?]
Think of an MDP as a mathematical description of a decision-making situation where:
\begin{itemize}
    \item You observe the current situation (state)
    \item You choose an action based on what you observe
    \item The world responds by transitioning to a new state and giving you a reward
    \item This process repeats over time
\end{itemize}
The key insight is that the future only depends on the current state, not the entire history—this is the Markov property.
\end{intuitionbox}

Markov Decision Processes provide the mathematical framework for modeling sequential decision-making under uncertainty. This chapter develops the formal theory of MDPs with particular attention to mathematical rigor and engineering applications.

\section{Understanding MDPs Through Examples}

Before diving into formal definitions, let's build intuition through a concrete example.

\begin{examplebox}[Grid World Navigation]
Consider a robot navigating a $4 \times 4$ grid world:
\begin{itemize}
    \item \textbf{States}: Each cell in the grid (16 total states)
    \item \textbf{Actions}: Move up, down, left, or right
    \item \textbf{Transitions}: Move to adjacent cell (or stay put if hitting a wall)
    \item \textbf{Rewards}: +10 for reaching the goal, -1 for each step, -10 for falling into holes
    \item \textbf{Goal}: Find the shortest path to the target while avoiding obstacles
\end{itemize}
\end{examplebox}

\section{Formal Definition and Mathematical Properties}

Now that we have intuition, let's formalize these concepts.

\begin{definition}[Markov Decision Process]
A Markov Decision Process is a 5-tuple $(\state, \action, \transition, \reward, \discount)$ where:
\begin{itemize}
    \item $\state$ is the \textbf{state space} (all possible situations)
    \item $\action$ is the \textbf{action space} (all possible actions)
    \item $\transition: \state \times \action \times \state \to [0,1]$ is the \textbf{transition kernel} (dynamics)
    \item $\reward: \state \times \action \to \real$ is the \textbf{reward function} (immediate feedback)
    \item $\discount \in [0,1)$ is the \textbf{discount factor} (how much we value future rewards)
\end{itemize}
\end{definition}

\begin{intuitionbox}[Understanding the Components]
\begin{itemize}
    \item \textbf{State space $\state$}: All possible configurations of your system
    \item \textbf{Action space $\action$}: All decisions you can make in any given state
    \item \textbf{Transition function $\transition$}: Describes how actions change states (the "physics" of your world)
    \item \textbf{Reward function $\reward$}: Immediate feedback telling you how good an action was
    \item \textbf{Discount factor $\discount$}: How much you care about future vs. immediate rewards (0 = only care about immediate, close to 1 = care about long-term)
\end{itemize}
\end{intuitionbox}

The transition kernel satisfies $\sum_{s' \in \state} \transition(s,a,s') = 1$ for all $(s,a) \in \state \times \action$, and we write $\transition(s'|s,a) = \transition(s,a,s')$ for the probability of transitioning to state $s'$ from state $s$ under action $a$.

\subsection{Assumptions and Regularity Conditions}

For mathematical tractability, we typically assume:

\begin{assumption}[Measurability]
The state and action spaces are measurable spaces, and the transition kernel and reward function are measurable with respect to the appropriate $\sigma$-algebras.
\end{assumption}

\begin{assumption}[Bounded Rewards]
The reward function satisfies $\sup_{s,a} |\reward(s,a)| \leq R_{max} < \infty$.
\end{assumption}

\begin{assumption}[Discount Factor]
The discount factor satisfies $\discount \in [0,1)$ to ensure convergence of infinite-horizon value functions.
\end{assumption}

\subsection{State and Action Spaces}

\subsubsection{Discrete Spaces}

For finite MDPs with $|\state| = n$ and $|\action| = m$, we can represent:
\begin{itemize}
    \item Transition probabilities as tensors $P^a \in \real^{n \times n}$ for each action $a$
    \item Rewards as matrices $R \in \real^{n \times m}$
    \item Policies as matrices $\Pi \in [0,1]^{n \times m}$ with $\sum_a \Pi(s,a) = 1$
\end{itemize}

\subsubsection{Continuous Spaces}

For continuous state spaces $\state \subseteq \real^d$, the transition kernel becomes a probability measure:
\begin{equation}
\transition(\cdot|s,a): \mathcal{B}(\state) \to [0,1]
\end{equation}
where $\mathcal{B}(\state)$ is the Borel $\sigma$-algebra on $\state$.

\begin{examplebox}[Engineering Example: Inverted Pendulum]
Consider an inverted pendulum with:
\begin{itemize}
    \item State: $s = (\theta, \dot{\theta}) \in [-\pi, \pi] \times [-10, 10]$ (angle and angular velocity)
    \item Action: $a \in [-5, 5]$ (applied torque)
    \item Dynamics: $\ddot{\theta} = \frac{g}{l}\sin\theta + \frac{a}{ml^2}$ plus noise
    \item Reward: $r(s,a) = -\theta^2 - 0.1\dot{\theta}^2 - 0.01a^2$ (quadratic cost)
\end{itemize}
\end{examplebox}

\section{Policies and Value Functions}

\begin{intuitionbox}[What is a Policy?]
A policy is simply a decision-making rule. It tells an agent what action to take in each possible state. Think of it as a strategy or game plan.
\end{intuitionbox}

\subsection{Types of Policies}

\begin{definition}[Deterministic Policy]
A deterministic policy is a function $\policy: \state \to \action$ that maps each state to exactly one action.
\end{definition}

\begin{examplebox}[Deterministic Policy Example]
In our grid world: "Always move towards the goal" could be a deterministic policy where $\policy(\text{state}) = \text{direction\_to\_goal}$.
\end{examplebox}

\begin{definition}[Stochastic Policy]
A stochastic policy $\policy: \state \to \Delta(\action)$ assigns a probability distribution over actions for each state, where $\Delta(\action)$ is the space of probability measures on $\action$.
\end{definition}

\begin{examplebox}[Stochastic Policy Example]
In grid world: "Move towards goal with probability 0.8, move randomly otherwise" gives $\policy(\text{best\_action}|s) = 0.8$ and equal probability to other actions.
\end{examplebox}

\begin{definition}[History-Dependent Policy]
A history-dependent policy depends on the entire sequence of past states and actions:
$\policy_t: (\state \times \action)^t \times \state \to \Delta(\action)$
\end{definition}

\begin{remarkbox}[Why Focus on Markovian Policies?]
While policies could potentially use the entire history, the Markov property means that optimal policies only need to depend on the current state. This greatly simplifies our analysis!
\end{remarkbox}

\begin{theorem}[Sufficiency of Markovian Policies]
For any history-dependent policy, there exists a Markovian policy that achieves the same expected discounted reward.
\end{theorem}

\begin{proof}
This follows from the Markov property of the state transitions. The expected future reward depends only on the current state, not the history of how that state was reached.
\end{proof}

\subsection{Value Function Theory}

\begin{definition}[State Value Function]
The state value function for policy $\policy$ is:
\begin{equation}
\valuefunction^\policy(s) = \expect^\policy\left[\sum_{t=0}^\infty \discount^t \reward(S_t, A_t) \mid S_0 = s\right]
\end{equation}
\end{definition}

\begin{definition}[Action Value Function]
The action value function (Q-function) for policy $\policy$ is:
\begin{equation}
\qvalue^\policy(s,a) = \expect^\policy\left[\sum_{t=0}^\infty \discount^t \reward(S_t, A_t) \mid S_0 = s, A_0 = a\right]
\end{equation}
\end{definition}

\begin{theorem}[Existence and Uniqueness of Value Functions]
Under Assumptions 1-3, the value functions $\valuefunction^\policy$ and $\qvalue^\policy$ exist, are unique, and satisfy $\|\valuefunction^\policy\|_\infty \leq \frac{R_{max}}{1-\discount}$.
\end{theorem}

\begin{proof}
The geometric series $\sum_{t=0}^\infty \discount^t R_{max}$ converges to $\frac{R_{max}}{1-\discount}$ since $\discount < 1$. Uniqueness follows from the linearity of expectation.
\end{proof}

\subsection{Bellman Equations}

The fundamental recursive relationships in reinforcement learning are the Bellman equations.

\begin{theorem}[Bellman Equations for Policy Evaluation]
For any policy $\policy$:
\begin{align}
\valuefunction^\policy(s) &= \sum_{a \in \action} \policy(a|s) \left[\reward(s,a) + \discount \sum_{s' \in \state} \transition(s'|s,a) \valuefunction^\policy(s')\right] \\
\qvalue^\policy(s,a) &= \reward(s,a) + \discount \sum_{s' \in \state} \transition(s'|s,a) \sum_{a' \in \action} \policy(a'|s') \qvalue^\policy(s',a')
\end{align}
\end{theorem}

\begin{proof}
By the tower rule of conditional expectation:
\begin{align}
\valuefunction^\policy(s) &= \expect^\policy\left[\reward(S_0, A_0) + \discount \sum_{t=1}^\infty \discount^{t-1} \reward(S_t, A_t) \mid S_0 = s\right] \\
&= \expect^\policy[\reward(S_0, A_0) | S_0 = s] + \discount \expect^\policy\left[\valuefunction^\policy(S_1) \mid S_0 = s\right]
\end{align}
Expanding the expectations gives the Bellman equation.
\end{proof}

\section{Optimal Policies and Bellman Optimality}

\subsection{Partial Ordering on Policies}

\begin{definition}[Policy Partial Order]
Policy $\policy_1$ dominates policy $\policy_2$ (written $\policy_1 \geq \policy_2$) if:
\begin{equation}
\valuefunction^{\policy_1}(s) \geq \valuefunction^{\policy_2}(s) \quad \forall s \in \state
\end{equation}
\end{definition}

\begin{theorem}[Existence of Optimal Policies]
There exists an optimal deterministic policy $\policy^*$ such that:
\begin{equation}
\valuefunction^{\policy^*}(s) = \max_\policy \valuefunction^\policy(s) \equiv \valuefunction^*(s) \quad \forall s \in \state
\end{equation}
\end{theorem}

\subsection{Bellman Optimality Equations}

\begin{theorem}[Bellman Optimality Equations]
The optimal value functions satisfy:
\begin{align}
\valuefunction^*(s) &= \max_{a \in \action} \left[\reward(s,a) + \discount \sum_{s' \in \state} \transition(s'|s,a) \valuefunction^*(s')\right] \\
\qvalue^*(s,a) &= \reward(s,a) + \discount \sum_{s' \in \state} \transition(s'|s,a) \max_{a' \in \action} \qvalue^*(s',a')
\end{align}
\end{theorem}

\begin{corollary}[Optimal Policy Extraction]
An optimal policy can be extracted from the optimal value functions as:
\begin{equation}
\policy^*(s) \in \argmax_{a \in \action} \qvalue^*(s,a)
\end{equation}
\end{corollary}

\section{Contraction Mapping Theorem and Fixed Points}

The mathematical foundation for proving convergence of dynamic programming algorithms relies on contraction mapping theory.

\subsection{Bellman Operators}

\begin{definition}[Bellman Operator]
For policy $\policy$, the Bellman operator $T^\policy: \real^\state \to \real^\state$ is defined by:
\begin{equation}
(T^\policy V)(s) = \sum_{a \in \action} \policy(a|s) \left[\reward(s,a) + \discount \sum_{s' \in \state} \transition(s'|s,a) V(s')\right]
\end{equation}
\end{definition}

\begin{definition}[Bellman Optimality Operator]
The Bellman optimality operator $T^*: \real^\state \to \real^\state$ is defined by:
\begin{equation}
(T^* V)(s) = \max_{a \in \action} \left[\reward(s,a) + \discount \sum_{s' \in \state} \transition(s'|s,a) V(s')\right]
\end{equation}
\end{definition}

\subsection{Contraction Properties}

\begin{theorem}[Contraction Property of Bellman Operators]
Under the supremum norm $\|V\|_\infty = \max_{s \in \state} |V(s)|$:
\begin{enumerate}
    \item $T^\policy$ is a $\discount$-contraction: $\|T^\policy V_1 - T^\policy V_2\|_\infty \leq \discount \|V_1 - V_2\|_\infty$
    \item $T^*$ is a $\discount$-contraction: $\|T^* V_1 - T^* V_2\|_\infty \leq \discount \|V_1 - V_2\|_\infty$
\end{enumerate}
\end{theorem}

\begin{proof}
For the policy operator:
\begin{align}
|(T^\policy V_1)(s) - (T^\policy V_2)(s)| &= \left|\sum_{a} \policy(a|s) \discount \sum_{s'} \transition(s'|s,a) [V_1(s') - V_2(s')]\right| \\
&\leq \sum_{a} \policy(a|s) \discount \sum_{s'} \transition(s'|s,a) |V_1(s') - V_2(s')| \\
&\leq \discount \|V_1 - V_2\|_\infty \sum_{a} \policy(a|s) \sum_{s'} \transition(s'|s,a) \\
&= \discount \|V_1 - V_2\|_\infty
\end{align}
The proof for $T^*$ follows similarly using the fact that the max operator is non-expansive.
\end{proof}

\subsection{Banach Fixed Point Theorem Application}

\begin{theorem}[Banach Fixed Point Theorem]
Let $(X, d)$ be a complete metric space and $T: X \to X$ be a contraction mapping with contraction factor $\gamma < 1$. Then:
\begin{enumerate}
    \item $T$ has a unique fixed point $x^* \in X$
    \item For any $x_0 \in X$, the sequence $x_{n+1} = T(x_n)$ converges to $x^*$
    \item The convergence rate is geometric: $d(x_n, x^*) \leq \gamma^n d(x_0, x^*)$
\end{enumerate}
\end{theorem}

\begin{corollary}[Convergence of Value Iteration]
The value iteration algorithm $V_{k+1} = T^* V_k$ converges geometrically to the unique optimal value function $\valuefunction^*$ at rate $\discount$.
\end{corollary}

\section{Policy Improvement and Optimality}

\subsection{Policy Improvement Theorem}

\begin{theorem}[Policy Improvement Theorem]
Let $\policy$ be any policy and define the improved policy $\policy'$ by:
\begin{equation}
\policy'(s) \in \argmax_{a \in \action} \qvalue^\policy(s,a)
\end{equation}
Then $\policy' \geq \policy$, with strict inequality unless $\policy$ is optimal.
\end{theorem}

\begin{proof}
For any state $s$:
\begin{align}
\qvalue^\policy(s, \policy'(s)) &\geq \qvalue^\policy(s, \policy(s)) = \valuefunction^\policy(s)
\end{align}
By the policy evaluation equation and induction, this implies $\valuefunction^{\policy'}(s) \geq \valuefunction^\policy(s)$.
\end{proof}

\subsection{Policy Iteration Algorithm}

The policy improvement theorem leads to the policy iteration algorithm:

\begin{algorithm}
\caption{Policy Iteration}
\begin{algorithmic}
\REQUIRE Initial policy $\policy_0$
\ENSURE Optimal policy $\policy^*$
\STATE $k \leftarrow 0$
\REPEAT
\STATE \textbf{Policy Evaluation:} Solve $\valuefunction^{\policy_k} = T^{\policy_k} \valuefunction^{\policy_k}$
\STATE \textbf{Policy Improvement:} $\policy_{k+1}(s) \leftarrow \argmax_a \qvalue^{\policy_k}(s,a)$
\STATE $k \leftarrow k + 1$
\UNTIL{$\policy_k = \policy_{k-1}$}
\RETURN $\policy^* = \policy_k$
\end{algorithmic}
\end{algorithm}

\begin{theorem}[Convergence of Policy Iteration]
Policy iteration converges to an optimal policy in finitely many iterations for finite MDPs.
\end{theorem}

\section{Computational Complexity Analysis}

\subsection{Value Iteration Complexity}

For finite MDPs with $|\state| = n$ and $|\action| = m$:

\begin{itemize}
    \item \textbf{Time per iteration:} $O(mn^2)$ operations
    \item \textbf{Iterations to $\epsilon$-accuracy:} $O(\log(\epsilon^{-1}))$ iterations
    \item \textbf{Total complexity:} $O(mn^2 \log(\epsilon^{-1}))$
\end{itemize}

\subsection{Policy Iteration Complexity}

\begin{itemize}
    \item \textbf{Policy evaluation:} $O(n^3)$ for direct matrix inversion, $O(n^2)$ per iteration for iterative methods
    \item \textbf{Policy improvement:} $O(mn^2)$
    \item \textbf{Number of policy iterations:} At most $m^n$ (typically much smaller)
\end{itemize}

\subsection{Modified Policy Iteration}

To balance the computational costs, modified policy iteration performs only $k$ steps of policy evaluation:

\begin{algorithm}
\caption{Modified Policy Iteration}
\begin{algorithmic}
\REQUIRE Initial policy $\policy_0$, evaluation steps $k$
\STATE Initialize $V_0$ arbitrarily
\FOR{$i = 0, 1, 2, \ldots$}
    \FOR{$j = 1, 2, \ldots, k$}
        \STATE $V_j \leftarrow T^{\policy_i} V_{j-1}$
    \ENDFOR
    \STATE $\policy_{i+1}(s) \leftarrow \argmax_a \left[r(s,a) + \gamma \sum_{s'} P(s'|s,a) V_k(s')\right]$
\ENDFOR
\end{algorithmic}
\end{algorithm}

\section{Connections to Classical Control Theory}

\subsection{Linear Quadratic Regulator (LQR)}

For linear dynamics $s_{t+1} = As_t + Ba_t + w_t$ and quadratic costs $r(s,a) = -s^TQs - a^TRa$, the optimal value function is quadratic: $\valuefunction^*(s) = -s^TPs$ where $P$ satisfies the discrete algebraic Riccati equation:

\begin{equation}
P = Q + A^TPA - A^TPB(R + B^TPB)^{-1}B^TPA
\end{equation}

The optimal policy is linear: $\policy^*(s) = -Ks$ where $K = (R + B^TPB)^{-1}B^TPA$.

\subsection{Hamilton-Jacobi-Bellman Equation}

For continuous-time systems, the Bellman equation becomes the Hamilton-Jacobi-Bellman (HJB) partial differential equation:

\begin{equation}
\frac{\partial V}{\partial t} + \min_a \left[r(s,a) + \frac{\partial V}{\partial s} f(s,a)\right] = 0
\end{equation}

where $f(s,a)$ is the system dynamics.

\section{Advanced Topics}

\subsection{Partially Observable MDPs (POMDPs)}

\begin{definition}[POMDP]
A POMDP extends an MDP with observations: $(\state, \action, \mathcal{O}, \transition, \reward, \mathcal{Z}, \discount)$ where:
\begin{itemize}
    \item $\mathcal{O}$ is the observation space
    \item $\mathcal{Z}: \state \times \action \times \mathcal{O} \to [0,1]$ is the observation model
\end{itemize}
\end{definition}

The optimal policy depends on the belief state $b(s) = \prob(S_t = s | h_t)$ where $h_t$ is the history of observations.

\subsection{Constrained MDPs}

\begin{definition}[Constrained MDP]
A constrained MDP adds constraint functions $c_i: \state \times \action \to \real$ and thresholds $d_i$:
\begin{align}
\max_\policy \quad &\expect^\policy\left[\sum_{t=0}^\infty \discount^t \reward(S_t, A_t)\right] \\
\text{subject to} \quad &\expect^\policy\left[\sum_{t=0}^\infty \discount^t c_i(S_t, A_t)\right] \leq d_i, \quad i = 1, \ldots, m
\end{align}
\end{definition}

Solutions typically use Lagrangian methods and primal-dual algorithms.

\section{Chapter Summary}

This chapter established the mathematical foundations of Markov Decision Processes:

\begin{itemize}
    \item Formal definition of MDPs and regularity assumptions
    \item Policy and value function theory with existence and uniqueness results
    \item Bellman equations and optimality conditions
    \item Contraction mapping theory and convergence guarantees
    \item Dynamic programming algorithms: value iteration and policy iteration
    \item Computational complexity analysis
    \item Connections to classical control theory and advanced extensions
\end{itemize}

The mathematical framework developed here provides the foundation for all reinforcement learning algorithms. The next chapter examines dynamic programming methods in detail, providing the algorithmic foundation for modern RL techniques.
\chapter{Dynamic Programming Foundations}
\label{ch:dynamic-programming}

Dynamic programming provides the theoretical and algorithmic foundation for reinforcement learning. This chapter develops the mathematical theory of dynamic programming with emphasis on convergence analysis, computational complexity, and connections to classical optimal control.

\section{Principle of Optimality}

The fundamental insight underlying dynamic programming is Bellman's principle of optimality, which enables the decomposition of complex sequential decision problems into simpler subproblems.

\begin{theorem}[Principle of Optimality]
An optimal policy has the property that whatever the initial state and initial decision are, the remaining decisions must constitute an optimal policy with regard to the state resulting from the first decision.
\end{theorem}

\subsection{Mathematical Formulation}

For an MDP $(\state, \action, \transition, \reward, \discount)$, consider a finite-horizon problem with horizon $T$. Define the optimal value function:

\begin{equation}
V_t^*(s) = \max_{\pi} \expect\left[\sum_{k=t}^{T-1} \discount^{k-t} \reward(S_k, A_k) \mid S_t = s, \pi\right]
\end{equation}

\begin{theorem}[Finite-Horizon Optimality]
The optimal value function satisfies the recursive relation:
\begin{align}
V_T^*(s) &= 0 \quad \forall s \in \state \\
V_t^*(s) &= \max_{a \in \action} \left[\reward(s,a) + \discount \sum_{s' \in \state} \transition(s'|s,a) V_{t+1}^*(s')\right]
\end{align}
for $t = T-1, T-2, \ldots, 0$.
\end{theorem}

\begin{proof}
The proof follows by backward induction. At time $T$, no more rewards can be collected, so $V_T^*(s) = 0$. For $t < T$, any optimal policy must choose the action that maximizes immediate reward plus discounted future value, leading to the recursive formulation.
\end{proof}

\subsection{Engineering Interpretation}

The principle of optimality has direct parallels in engineering optimization:

\begin{examplebox}[Optimal Control Example]
Consider a spacecraft trajectory optimization problem:
\begin{itemize}
    \item State: position and velocity $(x, v) \in \real^6$
    \item Control: thrust vector $u \in \real^3$
    \item Dynamics: $\dot{x} = v$, $\dot{v} = u/m - \nabla \Phi(x)$ (gravitational field)
    \item Cost: fuel consumption $\int_0^T \|u(t)\| dt$
\end{itemize}

The principle of optimality implies that if we have an optimal trajectory from Earth to Mars, then any sub-trajectory (e.g., from lunar orbit to Mars) must also be optimal for the sub-problem.
\end{examplebox}

\section{Value Iteration: Convergence Analysis}

Value iteration is the most fundamental algorithm in dynamic programming, providing a constructive method for computing optimal value functions.

\subsection{Algorithm Description}

\begin{algorithm}
\caption{Value Iteration}
\begin{algorithmic}
\REQUIRE MDP $(\state, \action, \transition, \reward, \discount)$, tolerance $\epsilon > 0$
\ENSURE $\epsilon$-optimal value function $V$
\STATE Initialize $V_0(s)$ arbitrarily for all $s \in \state$
\STATE $k \leftarrow 0$
\REPEAT
    \FOR{each $s \in \state$}
        \STATE $V_{k+1}(s) \leftarrow \max_{a \in \action} \left[\reward(s,a) + \discount \sum_{s' \in \state} \transition(s'|s,a) V_k(s')\right]$
    \ENDFOR
    \STATE $k \leftarrow k + 1$
\UNTIL{$\|V_k - V_{k-1}\|_\infty < \epsilon(1-\discount)/(2\discount)$}
\RETURN $V_k$
\end{algorithmic}
\end{algorithm}

\subsection{Convergence Theory}

\begin{theorem}[Convergence of Value Iteration]
For any initial value function $V_0$, the value iteration sequence $\{V_k\}_{k=0}^\infty$ defined by $V_{k+1} = T^* V_k$ converges to the unique optimal value function $V^*$ at geometric rate $\discount$.

Specifically:
\begin{equation}
\|V_k - V^*\|_\infty \leq \discount^k \|V_0 - V^*\|_\infty
\end{equation}
\end{theorem}

\begin{proof}
Since $T^*$ is a $\discount$-contraction in the supremum norm and $V^*$ is the unique fixed point of $T^*$, the result follows directly from the Banach fixed point theorem.
\end{proof}

\subsection{Error Bounds and Stopping Criteria}

\begin{theorem}[Error Bounds for Value Iteration]
If $\|V_{k+1} - V_k\|_\infty \leq \delta$, then:
\begin{align}
\|V_k - V^*\|_\infty &\leq \frac{\discount \delta}{1 - \discount} \\
\|V_{k+1} - V^*\|_\infty &\leq \frac{\delta}{1 - \discount}
\end{align}
\end{theorem}

\begin{proof}
Using the triangle inequality and contraction property:
\begin{align}
\|V_k - V^*\|_\infty &= \|V_k - T^* V_k + T^* V_k - V^*\|_\infty \\
&\leq \|V_k - T^* V_k\|_\infty + \|T^* V_k - T^* V^*\|_\infty \\
&= \|V_k - V_{k+1}\|_\infty + \discount \|V_k - V^*\|_\infty
\end{align}
Solving for $\|V_k - V^*\|_\infty$ gives the first bound. The second follows similarly.
\end{proof}

\begin{corollary}[Practical Stopping Criterion]
To achieve $\|V_k - V^*\|_\infty \leq \epsilon$, it suffices to stop when:
\begin{equation}
\|V_{k+1} - V_k\|_\infty \leq \epsilon(1 - \discount)
\end{equation}
\end{corollary}

\subsection{Computational Complexity}

\begin{theorem}[Sample Complexity of Value Iteration]
To achieve $\epsilon$-optimal value function, value iteration requires:
\begin{equation}
O\left(\frac{\log(\epsilon^{-1}) + \log(\|V_0 - V^*\|_\infty)}{1 - \discount}\right)
\end{equation}
iterations.
\end{theorem}

For each iteration:
\begin{itemize}
    \item \textbf{Time complexity:} $O(|\state|^2 |\action|)$ for tabular case
    \item \textbf{Space complexity:} $O(|\state|)$ for storing value function
    \item \textbf{Total operations:} $O(|\state|^2 |\action| \log(\epsilon^{-1}) / (1-\discount))$
\end{itemize}

\section{Policy Iteration: Mathematical Guarantees}

Policy iteration alternates between policy evaluation and policy improvement, providing an alternative approach with different computational characteristics.

\subsection{Policy Evaluation}

Given policy $\pi$, policy evaluation solves the linear system:
\begin{equation}
V^\pi = T^\pi V^\pi
\end{equation}

In matrix form for finite MDPs:
\begin{equation}
V^\pi = R^\pi + \discount P^\pi V^\pi
\end{equation}

where $R^\pi \in \real^{|\state|}$ and $P^\pi \in \real^{|\state| \times |\state|}$ are policy-specific reward and transition matrices.

\begin{theorem}[Unique Solution to Policy Evaluation]
The linear system $(I - \discount P^\pi) V^\pi = R^\pi$ has a unique solution:
\begin{equation}
V^\pi = (I - \discount P^\pi)^{-1} R^\pi
\end{equation}
since $\rho(P^\pi) \leq 1$ and $\discount < 1$ ensure $(I - \discount P^\pi)$ is invertible.
\end{theorem}

\subsection{Iterative Policy Evaluation}

For large state spaces, direct matrix inversion is computationally prohibitive. Iterative policy evaluation uses:
\begin{equation}
V_{k+1}^\pi = T^\pi V_k^\pi
\end{equation}

\begin{theorem}[Convergence of Iterative Policy Evaluation]
The sequence $\{V_k^\pi\}$ converges geometrically to $V^\pi$ at rate $\discount$:
\begin{equation}
\|V_k^\pi - V^\pi\|_\infty \leq \discount^k \|V_0^\pi - V^\pi\|_\infty
\end{equation}
\end{theorem}

\subsection{Policy Improvement Analysis}

\begin{theorem}[Strict Improvement or Optimality]
Given policy $\pi$ and improved policy $\pi'$ defined by:
\begin{equation}
\pi'(s) \in \argmax_{a \in \action} Q^\pi(s,a)
\end{equation}

Then either:
\begin{enumerate}
    \item $V^{\pi'}(s) > V^\pi(s)$ for some $s \in \state$ (strict improvement), or
    \item $V^{\pi'}(s) = V^\pi(s)$ for all $s \in \state$ (optimality)
\end{enumerate}
\end{theorem}

\begin{proof}
By construction, $Q^\pi(s, \pi'(s)) \geq Q^\pi(s, \pi(s)) = V^\pi(s)$ for all $s$. If inequality is strict for any state, then by the policy evaluation equations, strict improvement propagates. If equality holds everywhere, then $\pi$ satisfies the Bellman optimality equation and is optimal.
\end{proof}

\subsection{Global Convergence}

\begin{theorem}[Finite Convergence of Policy Iteration]
For finite MDPs, policy iteration converges to an optimal policy in finitely many iterations. Specifically, the number of iterations is bounded by $|\action|^{|\state|}$.
\end{theorem}

\begin{proof}
Since each iteration either strictly improves the policy or terminates at optimality, and there are finitely many deterministic policies, convergence must occur in finite time. The bound follows from counting the total number of deterministic policies.
\end{proof}

\section{Modified Policy Iteration}

Modified policy iteration interpolates between value iteration and policy iteration, providing computational flexibility.

\subsection{Algorithm and Convergence}

\begin{algorithm}
\caption{Modified Policy Iteration}
\begin{algorithmic}
\REQUIRE Initial policy $\pi_0$, evaluation steps $m$
\STATE $i \leftarrow 0$
\REPEAT
    \STATE $V \leftarrow$ arbitrary initialization
    \FOR{$k = 1, 2, \ldots, m$}
        \STATE $V \leftarrow T^{\pi_i} V$
    \ENDFOR
    \STATE $\pi_{i+1}(s) \leftarrow \argmax_a [r(s,a) + \gamma \sum_{s'} P(s'|s,a) V(s')]$
    \STATE $i \leftarrow i + 1$
\UNTIL{convergence}
\end{algorithmic}
\end{algorithm}

\begin{theorem}[Convergence of Modified Policy Iteration]
Modified policy iteration with $m \geq 1$ evaluation steps converges to an optimal policy. The convergence rate depends on $m$:
\begin{itemize}
    \item $m = 1$: reduces to value iteration with rate $\discount$
    \item $m = \infty$: reduces to policy iteration with finite convergence
    \item $1 < m < \infty$: intermediate convergence rate
\end{itemize}
\end{theorem}

\subsection{Optimal Choice of Evaluation Steps}

The computational trade-off between evaluation and improvement can be optimized:

\begin{theorem}[Optimal Evaluation Steps]
For modified policy iteration, the optimal number of evaluation steps $m^*$ minimizes total computational cost:
\begin{equation}
m^* = \argmin_m \left[\text{cost per iteration} \times \text{number of iterations}\right]
\end{equation}

Under reasonable assumptions about computational costs, $m^* = O(\log(1/(1-\discount)))$.
\end{theorem}

\section{Asynchronous Dynamic Programming}

Traditional DP algorithms update all states synchronously. Asynchronous variants can offer computational advantages and theoretical insights.

\subsection{Gauss-Seidel Value Iteration}

\begin{algorithm}
\caption{Gauss-Seidel Value Iteration}
\begin{algorithmic}
\STATE Order states $s_1, s_2, \ldots, s_n$
\REPEAT
    \FOR{$i = 1, 2, \ldots, n$}
        \STATE $V(s_i) \leftarrow \max_a \left[r(s_i,a) + \gamma \sum_{j} P(s_j|s_i,a) V(s_j)\right]$
    \ENDFOR
\UNTIL{convergence}
\end{algorithmic}
\end{algorithm}

\begin{theorem}[Convergence of Gauss-Seidel Value Iteration]
Gauss-Seidel value iteration converges to the optimal value function. The convergence rate can be faster than standard value iteration due to more frequent updates.
\end{theorem}

\subsection{Prioritized Sweeping}

\begin{definition}[Bellman Error]
For state $s$ and value function $V$, the Bellman error is:
\begin{equation}
\delta(s) = \left|\max_a \left[r(s,a) + \gamma \sum_{s'} P(s'|s,a) V(s')\right] - V(s)\right|
\end{equation}
\end{definition}

Prioritized sweeping updates states in order of decreasing Bellman error, focusing computation on states where updates will have the largest impact.

\begin{algorithm}
\caption{Prioritized Sweeping}
\begin{algorithmic}
\STATE Initialize priority queue $\mathcal{Q}$ with all states
\WHILE{$\mathcal{Q}$ not empty}
    \STATE $s \leftarrow$ state with highest priority in $\mathcal{Q}$
    \STATE Update $V(s)$ using Bellman equation
    \STATE Remove $s$ from $\mathcal{Q}$
    \FOR{each predecessor $s'$ of $s$}
        \IF{Bellman error of $s'$ exceeds threshold}
            \STATE Add $s'$ to $\mathcal{Q}$ with updated priority
        \ENDIF
    \ENDFOR
\ENDWHILE
\end{algorithmic}
\end{algorithm}

\subsection{Real-Time Dynamic Programming}

Real-time DP focuses updates on states visited by a simulated or actual agent trajectory.

\begin{algorithm}
\caption{Real-Time Dynamic Programming}
\begin{algorithmic}
\STATE Initialize current state $s$
\REPEAT
    \STATE Update $V(s)$ using Bellman equation
    \STATE Choose action $a = \argmax_a Q(s,a)$
    \STATE Simulate or execute action: $s \leftarrow s'$ with probability $P(s'|s,a)$
\UNTIL{termination}
\end{algorithmic}
\end{algorithm}

\begin{theorem}[Convergence of RTDP]
Under appropriate exploration conditions, real-time DP converges to optimal values on the states reachable under the optimal policy.
\end{theorem}

\section{Linear Programming Formulation}

Dynamic programming problems can be formulated as linear programs, providing alternative solution methods and theoretical insights.

\subsection{Primal LP Formulation}

The optimal value function can be found by solving:
\begin{align}
\minimize_{V} \quad &\sum_{s \in \state} \alpha(s) V(s) \\
\text{subject to} \quad &V(s) \geq r(s,a) + \gamma \sum_{s' \in \state} P(s'|s,a) V(s') \quad \forall s,a
\end{align}

where $\alpha(s) > 0$ represents state weights.

\begin{theorem}[LP-DP Equivalence]
The optimal solution to the linear program equals the optimal value function $V^*$.
\end{theorem}

\subsection{Dual LP Formulation}

The dual problem involves finding an optimal state-action visitation measure:
\begin{align}
\maximize_{\mu} \quad &\sum_{s,a} \mu(s,a) r(s,a) \\
\text{subject to} \quad &\sum_a \mu(s,a) - \gamma \sum_{s',a'} \mu(s',a') P(s|s',a') = \alpha(s) \quad \forall s \\
&\mu(s,a) \geq 0 \quad \forall s,a
\end{align}

\begin{theorem}[Strong Duality]
Under mild conditions, strong duality holds between the primal and dual formulations, and complementary slackness conditions characterize optimal policies.
\end{theorem}

\section{Connections to Classical Control Theory}

\subsection{Discrete-Time Optimal Control}

Consider the discrete-time optimal control problem:
\begin{align}
\minimize \quad &\sum_{t=0}^{T-1} L(x_t, u_t) + L_T(x_T) \\
\text{subject to} \quad &x_{t+1} = f(x_t, u_t) + w_t \\
&u_t \in \mathcal{U}(x_t)
\end{align}

The dynamic programming solution gives the Hamilton-Jacobi-Bellman equation:
\begin{equation}
V_t(x) = \min_{u \in \mathcal{U}(x)} [L(x,u) + \expect[V_{t+1}(f(x,u) + w)]]
\end{equation}

\subsection{Stochastic Optimal Control}

For stochastic control systems $dx_t = f(x_t, u_t) dt + \sigma(x_t, u_t) dW_t$, the continuous-time HJB equation is:

\begin{equation}
\frac{\partial V}{\partial t} + \min_u \left[L(x,u) + \frac{\partial V}{\partial x} f(x,u) + \frac{1}{2} \text{tr}\left(\sigma(x,u)^T \frac{\partial^2 V}{\partial x^2} \sigma(x,u)\right)\right] = 0
\end{equation}

\subsection{Model Predictive Control (MPC)}

MPC can be viewed as approximate dynamic programming with receding horizon:

\begin{algorithm}
\caption{Model Predictive Control}
\begin{algorithmic}
\REPEAT
    \STATE Measure current state $x_t$
    \STATE Solve optimization problem over horizon $[t, t+H]$:
    \STATE $u_t^*, \ldots, u_{t+H-1}^* = \argmin \sum_{k=0}^{H-1} L(x_{t+k}, u_{t+k}) + L_H(x_{t+H})$
    \STATE Apply $u_t^*$ and advance to next time step
\UNTIL{termination}
\end{algorithmic}
\end{algorithm}

The connection to DP provides stability and performance guarantees for MPC under appropriate conditions.

\section{Computational Considerations}

\subsection{Curse of Dimensionality}

The computational complexity of DP algorithms scales exponentially with state space dimension:
\begin{itemize}
    \item Memory: $O(|\state|)$ for value function storage
    \item Computation: $O(|\state|^2 |\action|)$ per iteration
    \item For continuous spaces: requires discretization or function approximation
\end{itemize}

\subsection{Approximate Dynamic Programming}

To handle large state spaces, approximate DP uses function approximation:
\begin{equation}
V(s) \approx \sum_{i=1}^n w_i \phi_i(s)
\end{equation}

where $\{\phi_i\}$ are basis functions and $\{w_i\}$ are parameters.

\begin{theorem}[Error Propagation in Approximate DP]
If the approximation error is bounded by $\epsilon$ in supremum norm:
\begin{equation}
\|V - \hat{V}\|_\infty \leq \epsilon
\end{equation}
then the policy derived from $\hat{V}$ satisfies:
\begin{equation}
\|V^{\hat{\pi}} - V^*\|_\infty \leq \frac{2\gamma \epsilon}{(1-\gamma)^2}
\end{equation}
\end{theorem}

\section{Chapter Summary}

This chapter developed the mathematical foundations of dynamic programming:

\begin{itemize}
    \item Principle of optimality and recursive decomposition
    \item Value iteration: convergence theory, error bounds, complexity analysis
    \item Policy iteration: linear algebra formulation, finite convergence
    \item Modified policy iteration and computational trade-offs
    \item Asynchronous variants: Gauss-Seidel, prioritized sweeping, real-time DP
    \item Linear programming formulations and duality theory
    \item Connections to classical optimal control and MPC
    \item Computational challenges and approximate methods
\end{itemize}

These algorithmic foundations provide the basis for understanding modern reinforcement learning methods. The next chapter begins our exploration of learning algorithms that estimate value functions from experience rather than exact knowledge of the MDP.
\chapter{Monte Carlo Methods}
\label{ch:monte-carlo}

Monte Carlo methods form the foundation of model-free reinforcement learning, enabling value function estimation from sample episodes without requiring knowledge of the environment dynamics. This chapter develops the mathematical theory of Monte Carlo estimation in the RL context, with emphasis on convergence analysis and variance reduction techniques.

\section{Monte Carlo Estimation Theory}

Monte Carlo methods estimate expectations by sampling. In reinforcement learning, we use sample episodes to estimate value functions without requiring the transition probabilities or reward function.

\subsection{Basic Monte Carlo Principle}

Consider estimating the expectation $\expect[X]$ of random variable $X$. The Monte Carlo estimator uses $n$ independent samples $X_1, \ldots, X_n$:

\begin{equation}
\hat{\mu}_n = \frac{1}{n} \sum_{i=1}^n X_i
\end{equation}

\begin{theorem}[Strong Law of Large Numbers]
If $\expect[|X|] < \infty$, then $\hat{\mu}_n \to \expect[X]$ almost surely as $n \to \infty$.
\end{theorem}

\begin{theorem}[Central Limit Theorem]
If $\text{Var}(X) = \sigma^2 < \infty$, then:
\begin{equation}
\sqrt{n}(\hat{\mu}_n - \expect[X]) \xrightarrow{d} \mathcal{N}(0, \sigma^2)
\end{equation}
\end{theorem}

\subsection{Application to Value Function Estimation}

For policy $\pi$, the value function is:
\begin{equation}
V^\pi(s) = \expect^\pi\left[\sum_{t=0}^\infty \gamma^t R_{t+1} \mid S_0 = s\right]
\end{equation}

Monte Carlo estimation uses sample returns $G_t = \sum_{k=0}^\infty \gamma^k R_{t+k+1}$ from episodes starting in state $s$ to estimate $V^\pi(s)$.

\section{First-Visit vs. Every-Visit Methods}

\subsection{First-Visit Monte Carlo}

\begin{algorithm}
\caption{First-Visit Monte Carlo Policy Evaluation}
\begin{algorithmic}
\REQUIRE Policy $\pi$ to evaluate
\STATE Initialize $V(s) \in \real$ arbitrarily for all $s \in \mathcal{S}$
\STATE Initialize $Returns(s) \leftarrow$ empty list for all $s \in \mathcal{S}$
\REPEAT
    \STATE Generate episode following $\pi$: $S_0, A_0, R_1, S_1, A_1, R_2, \ldots, S_{T-1}, A_{T-1}, R_T$
    \STATE $G \leftarrow 0$
    \FOR{$t = T-1, T-2, \ldots, 0$}
        \STATE $G \leftarrow \gamma G + R_{t+1}$
        \IF{$S_t$ not appear in $S_0, S_1, \ldots, S_{t-1}$}
            \STATE Append $G$ to $Returns(S_t)$
            \STATE $V(S_t) \leftarrow$ average$(Returns(S_t))$
        \ENDIF
    \ENDFOR
\UNTIL{convergence}
\end{algorithmic}
\end{algorithm}

\subsection{Every-Visit Monte Carlo}

Every-visit MC updates the value estimate every time a state is visited in an episode, not just the first time.

\begin{theorem}[Convergence of First-Visit Monte Carlo]
First-visit Monte Carlo converges to $V^\pi(s)$ as the number of first visits to state $s$ approaches infinity, assuming:
\begin{enumerate}
    \item Episodes are generated according to policy $\pi$
    \item Each state has non-zero probability of being the starting state
    \item Returns have finite variance
\end{enumerate}
\end{theorem}

\begin{proof}
Each first visit to state $s$ provides an unbiased sample of the return. By the strong law of large numbers, the sample average converges to the true expectation.
\end{proof}

\begin{theorem}[Convergence of Every-Visit Monte Carlo]
Every-visit Monte Carlo also converges to $V^\pi(s)$ under similar conditions, despite the correlation between visits within the same episode.
\end{theorem}

\section{Variance Reduction Techniques}

\subsection{Incremental Implementation}

Instead of storing all returns, we can update estimates incrementally:

\begin{equation}
V_{n+1}(s) = V_n(s) + \frac{1}{n+1}[G_n - V_n(s)]
\end{equation}

More generally, with step size $\alpha$:
\begin{equation}
V(s) \leftarrow V(s) + \alpha[G - V(s)]
\end{equation}

\subsection{Baseline Subtraction}

To reduce variance, we can subtract a baseline $b(s)$ that doesn't depend on the action:

\begin{equation}
G_t - b(S_t)
\end{equation}

The optimal baseline that minimizes variance is:
\begin{equation}
b^*(s) = \frac{\expect[G_t^2 \mid S_t = s]}{\expect[G_t \mid S_t = s]} = \expect[G_t \mid S_t = s] = V^\pi(s)
\end{equation}

\subsection{Control Variates}

For correlated random variable $Y$ with known expectation $\expect[Y] = \mu_Y$:
\begin{equation}
\hat{\mu}_{CV} = \hat{\mu}_X - c(\hat{\mu}_Y - \mu_Y)
\end{equation}

The optimal coefficient is:
\begin{equation}
c^* = \frac{\text{Cov}(X,Y)}{\text{Var}(Y)}
\end{equation}

\section{Importance Sampling in RL}

Importance sampling enables off-policy learning by weighting samples according to the ratio of target to behavior policy probabilities.

\subsection{Ordinary Importance Sampling}

To estimate $\expect_\pi[X]$ using samples from policy $\mu$:
\begin{equation}
\hat{\mu}_{IS} = \frac{1}{n} \sum_{i=1}^n \rho_i X_i
\end{equation}

where $\rho_i = \frac{\pi(A_i|S_i)}{\mu(A_i|S_i)}$ is the importance sampling ratio.

\begin{theorem}[Unbiasedness of Importance Sampling]
$\expect[\hat{\mu}_{IS}] = \expect_\pi[X]$ if $\mu(a|s) > 0$ whenever $\pi(a|s) > 0$.
\end{theorem}

\subsection{Weighted Importance Sampling}

To reduce variance when some importance weights are very large:
\begin{equation}
\hat{\mu}_{WIS} = \frac{\sum_{i=1}^n \rho_i X_i}{\sum_{i=1}^n \rho_i}
\end{equation}

\begin{theorem}[Bias-Variance Tradeoff]
Weighted importance sampling is biased but often has lower variance than ordinary importance sampling:
\begin{align}
\text{Bias}[\hat{\mu}_{WIS}] &\neq 0 \text{ (in general)} \\
\text{Var}[\hat{\mu}_{WIS}] &\leq \text{Var}[\hat{\mu}_{IS}] \text{ (typically)}
\end{align}
\end{theorem}

\subsection{Per-Decision Importance Sampling}

For episodic tasks, the importance sampling ratio for a complete episode is:
\begin{equation}
\rho_{t:T-1} = \prod_{k=t}^{T-1} \frac{\pi(A_k|S_k)}{\mu(A_k|S_k)}
\end{equation}

This can have very high variance. Per-decision importance sampling uses only the relevant portion of the trajectory.

\section{Off-Policy Monte Carlo Methods}

\subsection{Off-Policy Policy Evaluation}

\begin{algorithm}
\caption{Off-Policy Monte Carlo Policy Evaluation}
\begin{algorithmic}
\REQUIRE Target policy $\pi$, behavior policy $\mu$
\STATE Initialize $V(s) \in \real$ arbitrarily for all $s \in \mathcal{S}$
\STATE Initialize $C(s) \leftarrow 0$ for all $s \in \mathcal{S}$
\REPEAT
    \STATE Generate episode using $\mu$: $S_0, A_0, R_1, \ldots, S_{T-1}, A_{T-1}, R_T$
    \STATE $G \leftarrow 0$
    \STATE $W \leftarrow 1$
    \FOR{$t = T-1, T-2, \ldots, 0$}
        \STATE $G \leftarrow \gamma G + R_{t+1}$
        \STATE $C(S_t) \leftarrow C(S_t) + W$
        \STATE $V(S_t) \leftarrow V(S_t) + \frac{W}{C(S_t)}[G - V(S_t)]$
        \STATE $W \leftarrow W \frac{\pi(A_t|S_t)}{\mu(A_t|S_t)}$
        \IF{$W = 0$}
            \STATE break
        \ENDIF
    \ENDFOR
\UNTIL{convergence}
\end{algorithmic}
\end{algorithm}

\subsection{Off-Policy Monte Carlo Control}

\begin{algorithm}
\caption{Off-Policy Monte Carlo Control}
\begin{algorithmic}
\STATE Initialize $Q(s,a) \in \real$ arbitrarily for all $s,a$
\STATE Initialize $C(s,a) \leftarrow 0$ for all $s,a$
\STATE Initialize $\pi(s) \leftarrow \argmax_a Q(s,a)$ for all $s$
\REPEAT
    \STATE Choose any soft policy $\mu$ (e.g., $\epsilon$-greedy)
    \STATE Generate episode using $\mu$
    \STATE $G \leftarrow 0$
    \STATE $W \leftarrow 1$
    \FOR{$t = T-1, T-2, \ldots, 0$}
        \STATE $G \leftarrow \gamma G + R_{t+1}$
        \STATE $C(S_t, A_t) \leftarrow C(S_t, A_t) + W$
        \STATE $Q(S_t, A_t) \leftarrow Q(S_t, A_t) + \frac{W}{C(S_t, A_t)}[G - Q(S_t, A_t)]$
        \STATE $\pi(S_t) \leftarrow \argmax_a Q(S_t, a)$
        \IF{$A_t \neq \pi(S_t)$}
            \STATE break
        \ENDIF
        \STATE $W \leftarrow W \frac{1}{\mu(A_t|S_t)}$
    \ENDFOR
\UNTIL{convergence}
\end{algorithmic}
\end{algorithm}

\section{Convergence Analysis and Sample Complexity}

\subsection{Finite Sample Analysis}

\begin{theorem}[Finite Sample Bound for Monte Carlo]
Let $V_n(s)$ be the Monte Carlo estimate after $n$ visits to state $s$. Under bounded rewards $|R| \leq R_{max}$:
\begin{equation}
\prob\left(|V_n(s) - V^\pi(s)| \geq \epsilon\right) \leq 2\exp\left(-\frac{2n\epsilon^2(1-\gamma)^2}{R_{max}^2}\right)
\end{equation}
\end{theorem}

\subsection{Asymptotic Convergence Rate}

\begin{theorem}[Central Limit Theorem for Monte Carlo]
If $\text{Var}^\pi[G_t | S_t = s] = \sigma^2(s) < \infty$, then:
\begin{equation}
\sqrt{n}(V_n(s) - V^\pi(s)) \xrightarrow{d} \mathcal{N}(0, \sigma^2(s))
\end{equation}
\end{theorem}

This gives the convergence rate $O(n^{-1/2})$, which is slower than the $O(n^{-1})$ rate achievable by temporal difference methods under certain conditions.

\subsection{Sample Complexity}

\begin{theorem}[Sample Complexity of Monte Carlo]
To achieve $\epsilon$-accurate value function estimation with probability $1-\delta$:
\begin{equation}
n \geq \frac{R_{max}^2 \log(2/\delta)}{2\epsilon^2(1-\gamma)^2}
\end{equation}
samples are sufficient.
\end{theorem}

\section{Practical Considerations}

\subsection{Exploration vs. Exploitation}

Monte Carlo control methods face the exploration-exploitation dilemma. Common approaches:

\textbf{Exploring Starts:} Assume episodes start in randomly selected state-action pairs.

\textbf{$\epsilon$-Greedy Policies:} Use soft policies that maintain exploration:
\begin{equation}
\pi(a|s) = \begin{cases}
1 - \epsilon + \frac{\epsilon}{|\mathcal{A}(s)|} & \text{if } a = \argmax_a Q(s,a) \\
\frac{\epsilon}{|\mathcal{A}(s)|} & \text{otherwise}
\end{cases}
\end{equation}

\subsection{Function Approximation}

For large state spaces, we approximate value functions:
\begin{equation}
V(s) \approx \hat{V}(s, \mathbf{w}) = \mathbf{w}^T \boldsymbol{\phi}(s)
\end{equation}

The Monte Carlo update becomes:
\begin{equation}
\mathbf{w} \leftarrow \mathbf{w} + \alpha[G_t - \hat{V}(S_t, \mathbf{w})]\nabla_\mathbf{w} \hat{V}(S_t, \mathbf{w})
\end{equation}

\begin{theorem}[Convergence with Linear Function Approximation]
Under linear function approximation with linearly independent features, Monte Carlo methods converge to the best linear approximation in the $L^2$ norm weighted by the stationary distribution.
\end{theorem}

\section{Chapter Summary}

This chapter established the foundations of Monte Carlo methods in reinforcement learning:

\begin{itemize}
    \item Monte Carlo estimation theory and convergence properties
    \item First-visit vs. every-visit methods with convergence guarantees
    \item Variance reduction techniques: baselines, control variates, importance sampling
    \item Off-policy learning through importance sampling with bias-variance analysis
    \item Sample complexity bounds and convergence rates
    \item Practical considerations for exploration and function approximation
\end{itemize}

Monte Carlo methods provide unbiased estimates and are conceptually simple, but they require complete episodes and have slower convergence than temporal difference methods. The next chapter develops temporal difference learning, which enables learning from individual transitions.
\chapter{Temporal Difference Learning}
\label{ch:temporal-difference}

Temporal Difference (TD) learning combines ideas from Monte Carlo methods and dynamic programming, enabling learning from incomplete episodes while maintaining the model-free nature of Monte Carlo methods. This chapter develops the mathematical theory of TD learning with emphasis on convergence analysis and the fundamental bias-variance tradeoff.

\section{TD(0) Algorithm and Mathematical Analysis}

\subsection{Basic TD(0) Update}

The core insight of temporal difference learning is to use the current estimate of the successor state's value to update the current state's value:

\begin{equation}
V(S_t) \leftarrow V(S_t) + \alpha [R_{t+1} + \gamma V(S_{t+1}) - V(S_t)]
\end{equation}

The TD error is defined as:
\begin{equation}
\delta_t = R_{t+1} + \gamma V(S_{t+1}) - V(S_t)
\end{equation}

\begin{algorithm}
\caption{Tabular TD(0) Policy Evaluation}
\begin{algorithmic}
\REQUIRE Policy $\pi$ to evaluate, step size $\alpha \in (0,1]$
\STATE Initialize $V(s) \in \real$ arbitrarily for all $s \in \mathcal{S}$, except $V(\text{terminal}) = 0$
\REPEAT
    \STATE Initialize $S$
    \REPEAT
        \STATE $A \leftarrow$ action given by $\pi$ for $S$
        \STATE Take action $A$, observe $R, S'$
        \STATE $V(S) \leftarrow V(S) + \alpha[R + \gamma V(S') - V(S)]$
        \STATE $S \leftarrow S'$
    \UNTIL{$S$ is terminal}
\UNTIL{convergence or sufficient accuracy}
\end{algorithmic}
\end{algorithm}

\subsection{Relationship to Bellman Equation}

The TD(0) update can be viewed as a stochastic approximation to the Bellman equation. The expected TD update is:

\begin{align}
\expect[\delta_t | S_t = s] &= \expect[R_{t+1} + \gamma V(S_{t+1}) - V(S_t) | S_t = s] \\
&= \sum_{s',r} p(s',r|s,\pi(s))[r + \gamma V(s') - V(s)] \\
&= (T^\pi V)(s) - V(s)
\end{align}

where $T^\pi$ is the Bellman operator for policy $\pi$.

\subsection{Convergence Analysis}

\begin{theorem}[Convergence of TD(0) - Tabular Case]
For the tabular case with appropriate step size sequence $\{\alpha_t\}$ satisfying:
\begin{align}
\sum_{t=0}^\infty \alpha_t &= \infty \\
\sum_{t=0}^\infty \alpha_t^2 &< \infty
\end{align}
TD(0) converges to $V^\pi$ with probability 1.
\end{theorem}

\begin{proof}[Proof Sketch]
The proof uses stochastic approximation theory. Define the ODE:
\begin{equation}
\frac{dV}{dt} = \expect[\delta_t | V] = T^\pi V - V
\end{equation}
Since $T^\pi$ is a contraction, the unique fixed point is $V^\pi$. The stochastic approximation theorem ensures convergence of the discrete updates to the ODE solution.
\end{proof}

\section{Bias-Variance Tradeoff in TD Methods}

\subsection{Bias Analysis}

TD(0) uses the biased estimate $R_{t+1} + \gamma V(S_{t+1})$ as a target for $V(S_t)$, while Monte Carlo uses the unbiased estimate $G_t$.

\begin{theorem}[Bias of TD Target]
The TD target $R_{t+1} + \gamma V(S_{t+1})$ has bias:
\begin{equation}
\text{Bias}[R_{t+1} + \gamma V(S_{t+1})] = \gamma[\hat{V}(S_{t+1}) - V^\pi(S_{t+1})]
\end{equation}
where $\hat{V}$ is the current estimate.
\end{theorem}

\subsection{Variance Analysis}

\begin{theorem}[Variance Comparison]
Under the assumption that value function errors are small, the variance of the TD target is approximately:
\begin{equation}
\text{Var}[R_{t+1} + \gamma V(S_{t+1})] \approx \text{Var}[R_{t+1}]
\end{equation}
while the Monte Carlo target has variance:
\begin{equation}
\text{Var}[G_t] = \text{Var}\left[\sum_{k=0}^\infty \gamma^k R_{t+k+1}\right]
\end{equation}
which is typically much larger.
\end{theorem}

\subsection{Mean Squared Error Decomposition}

\begin{equation}
\text{MSE} = \text{Bias}^2 + \text{Variance} + \text{Noise}
\end{equation}

TD methods trade increased bias for reduced variance, often resulting in lower overall MSE and faster convergence.

\section{TD(λ) and Eligibility Traces}

TD(λ) provides a family of algorithms that interpolate between TD(0) and Monte Carlo methods through the use of eligibility traces.

\subsection{Forward View: n-step Returns}

The n-step return combines rewards from the next n steps with the estimated value of the state reached after n steps:

\begin{equation}
G_t^{(n)} = R_{t+1} + \gamma R_{t+2} + \cdots + \gamma^{n-1} R_{t+n} + \gamma^n V(S_{t+n})
\end{equation}

The n-step TD update is:
\begin{equation}
V(S_t) \leftarrow V(S_t) + \alpha[G_t^{(n)} - V(S_t)]
\end{equation}

\subsection{λ-Return}

The λ-return combines all n-step returns:
\begin{equation}
G_t^\lambda = (1-\lambda) \sum_{n=1}^\infty \lambda^{n-1} G_t^{(n)}
\end{equation}

\begin{theorem}[λ-Return Properties]
The λ-return satisfies:
\begin{align}
G_t^\lambda &= R_{t+1} + \gamma[(1-\lambda)V(S_{t+1}) + \lambda G_{t+1}^\lambda] \\
\lim_{\lambda \to 0} G_t^\lambda &= R_{t+1} + \gamma V(S_{t+1}) \quad \text{(TD(0))} \\
\lim_{\lambda \to 1} G_t^\lambda &= G_t \quad \text{(Monte Carlo)}
\end{align}
\end{theorem}

\subsection{Backward View: Eligibility Traces}

Eligibility traces provide an online, incremental implementation of TD(λ):

\begin{align}
\delta_t &= R_{t+1} + \gamma V(S_{t+1}) - V(S_t) \\
e_t(s) &= \begin{cases}
\gamma \lambda e_{t-1}(s) + 1 & \text{if } s = S_t \\
\gamma \lambda e_{t-1}(s) & \text{if } s \neq S_t
\end{cases} \\
V(s) &\leftarrow V(s) + \alpha \delta_t e_t(s) \quad \forall s
\end{align}

\begin{algorithm}
\caption{TD(λ) with Eligibility Traces}
\begin{algorithmic}
\REQUIRE Policy $\pi$, step size $\alpha$, trace decay $\lambda$
\STATE Initialize $V(s) \in \real$ arbitrarily for all $s$
\REPEAT
    \STATE Initialize $S$, $e(s) = 0$ for all $s$
    \REPEAT
        \STATE $A \leftarrow$ action given by $\pi$ for $S$
        \STATE Take action $A$, observe $R, S'$
        \STATE $\delta \leftarrow R + \gamma V(S') - V(S)$
        \STATE $e(S) \leftarrow e(S) + 1$
        \FOR{all $s$}
            \STATE $V(s) \leftarrow V(s) + \alpha \delta e(s)$
            \STATE $e(s) \leftarrow \gamma \lambda e(s)$
        \ENDFOR
        \STATE $S \leftarrow S'$
    \UNTIL{$S$ is terminal}
\UNTIL{convergence}
\end{algorithmic}
\end{algorithm}

\subsection{Equivalence Theorem}

\begin{theorem}[Forward-Backward Equivalence]
Under certain conditions, the forward view (using λ-returns) and backward view (using eligibility traces) produce identical updates when applied offline to a complete episode.
\end{theorem}

\section{Convergence Theory for Linear Function Approximation}

When the state space is large, we use function approximation:
\begin{equation}
V(s) \approx \hat{V}(s, \mathbf{w}) = \mathbf{w}^T \boldsymbol{\phi}(s)
\end{equation}

The TD(0) update becomes:
\begin{equation}
\mathbf{w}_{t+1} = \mathbf{w}_t + \alpha[R_{t+1} + \gamma \mathbf{w}_t^T \boldsymbol{\phi}(S_{t+1}) - \mathbf{w}_t^T \boldsymbol{\phi}(S_t)]\boldsymbol{\phi}(S_t)
\end{equation}

\subsection{Projected Bellman Equation}

Under linear function approximation, TD(0) converges to the solution of the projected Bellman equation:
\begin{equation}
\mathbf{w}^* = \arg\min_\mathbf{w} \|\boldsymbol{\Phi}\mathbf{w} - T^\pi(\boldsymbol{\Phi}\mathbf{w})\|_{\mathbf{D}}^2
\end{equation}

where $\boldsymbol{\Phi}$ is the feature matrix and $\mathbf{D}$ is a diagonal matrix of state visitation probabilities.

\begin{theorem}[Convergence of Linear TD(0)]
Under linear function approximation, TD(0) converges to:
\begin{equation}
\mathbf{w}^* = (\boldsymbol{\Phi}^T \mathbf{D} \boldsymbol{\Phi})^{-1} \boldsymbol{\Phi}^T \mathbf{D} \mathbf{r}^\pi
\end{equation}
where $\mathbf{r}^\pi$ is the expected reward vector.
\end{theorem}

\subsection{Error Bounds}

\begin{theorem}[Approximation Error Bound]
Let $V^*$ be the optimal value function and $\hat{V}^*$ be the best linear approximation. Then:
\begin{equation}
\|V^\pi - \hat{V}^\pi\|_{\mathbf{D}} \leq \frac{1}{1-\gamma} \min_\mathbf{w} \|V^\pi - \boldsymbol{\Phi}\mathbf{w}\|_{\mathbf{D}}
\end{equation}
\end{theorem}

\section{Comparison with Monte Carlo and DP Methods}

\subsection{Computational Complexity}

\begin{center}
\begin{tabular}{lccc}
\toprule
Method & Memory & Computation per Step & Episode Completion \\
\midrule
DP & $O(|\mathcal{S}|^2|\mathcal{A}|)$ & $O(|\mathcal{S}|^2|\mathcal{A}|)$ & Not Required \\
MC & $O(|\mathcal{S}|)$ & $O(1)$ & Required \\
TD & $O(|\mathcal{S}|)$ & $O(1)$ & Not Required \\
\bottomrule
\end{tabular}
\end{center}

\subsection{Sample Efficiency}

\begin{theorem}[Sample Complexity Comparison]
Under certain regularity conditions:
\begin{itemize}
    \item TD methods: $O(\frac{1}{\epsilon^2(1-\gamma)^2})$ samples for $\epsilon$-accuracy
    \item MC methods: $O(\frac{1}{\epsilon^2(1-\gamma)^4})$ samples for $\epsilon$-accuracy
\end{itemize}
\end{theorem}

TD methods often have better sample efficiency due to lower variance, despite being biased.

\subsection{Bootstrapping vs. Sampling}

\textbf{Bootstrapping:} Using estimates of successor states (DP, TD)
\textbf{Sampling:} Using actual experience (MC, TD)

TD methods combine both, leading to:
\begin{itemize}
    \item Faster learning than MC (bootstrapping)
    \item Model-free nature (sampling)
    \item Online learning capability
\end{itemize}

\section{Advanced Topics}

\subsection{Multi-step Methods}

The n-step TD methods generalize between TD(0) and Monte Carlo:
\begin{equation}
V(S_t) \leftarrow V(S_t) + \alpha[G_t^{(n)} - V(S_t)]
\end{equation}

\begin{theorem}[Optimal Step Size]
For n-step methods, there exists an optimal n that minimizes mean squared error, typically $n \in [3, 10]$ for many problems.
\end{theorem}

\subsection{True Online TD(λ)}

The classical TD(λ) is not equivalent to the forward view when using function approximation. True online TD(λ) corrects this:

\begin{align}
\mathbf{w}_{t+1} &= \mathbf{w}_t + \alpha \delta_t \mathbf{z}_t + \alpha(\mathbf{w}_t^T \boldsymbol{\phi}_t - \mathbf{w}_{t-1}^T \boldsymbol{\phi}_t)(\mathbf{z}_t - \boldsymbol{\phi}_t) \\
\mathbf{z}_{t+1} &= \gamma \lambda \mathbf{z}_t + \boldsymbol{\phi}_{t+1} - \alpha \gamma \lambda (\mathbf{z}_t^T \boldsymbol{\phi}_{t+1})\boldsymbol{\phi}_{t+1}
\end{align}

\subsection{Gradient TD Methods}

To handle function approximation more rigorously, gradient TD methods minimize the mean squared projected Bellman error:

\begin{align}
\text{MSPBE}(\mathbf{w}) &= \|\boldsymbol{\Pi}(\mathbf{T}^\pi \hat{\mathbf{v}} - \hat{\mathbf{v}})\|_{\mathbf{D}}^2 \\
\nabla \text{MSPBE}(\mathbf{w}) &= 2\boldsymbol{\Phi}^T \mathbf{D} (\boldsymbol{\Pi}(\mathbf{T}^\pi \hat{\mathbf{v}} - \hat{\mathbf{v}}))
\end{align}

\section{Chapter Summary}

This chapter developed the mathematical foundations of temporal difference learning:

\begin{itemize}
    \item TD(0) algorithm with convergence analysis using stochastic approximation theory
    \item Bias-variance tradeoff analysis showing TD's advantage in variance reduction
    \item TD(λ) and eligibility traces providing a spectrum between TD(0) and Monte Carlo
    \item Convergence theory for linear function approximation with error bounds
    \item Comparative analysis with Monte Carlo and dynamic programming methods
    \item Advanced topics including multi-step methods and gradient TD approaches
\end{itemize}

Temporal difference learning provides the foundation for many modern RL algorithms, combining the best aspects of Monte Carlo and dynamic programming approaches. The next chapter extends these ideas to action-value methods with Q-learning and SARSA.

% Part II: Function Approximation
\part{Function Approximation}

This part extends reinforcement learning beyond tabular methods to handle large and continuous state spaces through function approximation. We explore advanced temporal difference methods, develop linear function approximation with rigorous convergence analysis, and culminate with neural network approaches that enable deep reinforcement learning.

The transition from exact to approximate methods introduces new challenges including the deadly triad of function approximation, bootstrapping, and off-policy learning. We address these challenges with careful algorithm design and theoretical analysis.

\chapter{Q-Learning and SARSA Extensions}
\label{ch:q-learning-extensions}

\begin{keyideabox}[Chapter Overview]
This chapter extends our understanding of temporal difference control by exploring advanced variations of Q-learning and SARSA. We examine multi-step methods, eligibility traces, and theoretical convergence guarantees for off-policy learning. The mathematical analysis includes detailed proofs of convergence conditions and performance bounds.
\end{keyideabox}

\begin{intuitionbox}[From Basic TD to Advanced Control]
While Chapter 5 introduced the fundamental concepts of TD learning, real-world applications require more sophisticated approaches. Think of basic Q-learning as learning to drive on a simple track - it works, but for complex scenarios like city driving, you need advanced techniques that can handle delayed rewards, partial observability, and efficient exploration.
\end{intuitionbox}

\section{Multi-Step Q-Learning}

\subsection{n-Step Q-Learning}

The basic Q-learning update uses only the immediate next reward and state. Multi-step methods extend this by looking ahead multiple steps:

\begin{equation}
Q_{t+n}(S_t, A_t) = Q_t(S_t, A_t) + \alpha_t \left[ G_{t:t+n} - Q_t(S_t, A_t) \right]
\end{equation}

where the n-step return is defined as:
\begin{equation}
G_{t:t+n} = R_{t+1} + \gamma R_{t+2} + \cdots + \gamma^{n-1} R_{t+n} + \gamma^n \max_a Q_t(S_{t+n}, a)
\end{equation}

\begin{algorithm}
\caption{n-Step Q-Learning}
\begin{algorithmic}
\REQUIRE Step size $\alpha \in (0,1]$, small $\epsilon > 0$, positive integer $n$
\STATE Initialize $Q(s,a)$ arbitrarily for all $s \in \mathcal{S}, a \in \mathcal{A}(s)$, except $Q(\text{terminal}, \cdot) = 0$
\STATE Initialize and store $S_0$, select and store an action $A_0 \sim \pi(\cdot|S_0)$
\FOR{$t = 0, 1, 2, \ldots$}
    \IF{$t < T$}
        \STATE Take action $A_t$, observe and store the next reward as $R_{t+1}$ and the next state as $S_{t+1}$
        \IF{$S_{t+1}$ is terminal}
            \STATE $T \leftarrow t + 1$
        \ELSE
            \STATE Select and store $A_{t+1} \sim \pi(\cdot|S_{t+1})$
        \ENDIF
    \ENDIF
    \STATE $\tau \leftarrow t - n + 1$ (the time whose state's estimate is being updated)
    \IF{$\tau \geq 0$}
        \STATE $G \leftarrow \sum_{i=\tau+1}^{\min(\tau+n, T)} \gamma^{i-\tau-1} R_i$
        \IF{$\tau + n < T$}
            \STATE $G \leftarrow G + \gamma^n \max_a Q(S_{\tau+n}, a)$
        \ENDIF
        \STATE $Q(S_\tau, A_\tau) \leftarrow Q(S_\tau, A_\tau) + \alpha [G - Q(S_\tau, A_\tau)]$
    \ENDIF
\ENDFOR
\end{algorithmic}
\end{algorithm}

\subsection{Theoretical Analysis of n-Step Methods}

\begin{theorem}[n-Step Q-Learning Convergence]
Under standard conditions (bounded rewards, decreasing step size satisfying $\sum_t \alpha_t = \infty$ and $\sum_t \alpha_t^2 < \infty$, and sufficient exploration), n-step Q-learning converges to the optimal action-value function $Q^*$ with probability 1.
\end{theorem}

\begin{proof}
The proof follows by showing that the n-step return is an unbiased estimate of the optimal value under the greedy policy, then applying the stochastic approximation convergence theorem.

Let $\pi_t$ be the greedy policy with respect to $Q_t$. The n-step return can be written as:
\begin{align}
G_{t:t+n} &= \expect_{\pi_t}[R_{t+1} + \gamma R_{t+2} + \cdots + \gamma^{n-1} R_{t+n}] \\
&\quad + \gamma^n \max_a Q_t(S_{t+n}, a) + \text{martingale terms}
\end{align}

As $Q_t \to Q^*$, the bias in this estimate vanishes, ensuring convergence.
\end{proof}

\section{Q($\lambda$) Learning}

\subsection{Eligibility Traces for Q-Learning}

Eligibility traces provide an efficient way to update all state-action pairs based on their recency and frequency of visitation:

\begin{equation}
e_t(s,a) = \begin{cases}
\gamma \lambda e_{t-1}(s,a) + 1 & \text{if } s = S_t \text{ and } a = A_t \\
\gamma \lambda e_{t-1}(s,a) & \text{otherwise}
\end{cases}
\end{equation}

The Q($\lambda$) update is then:
\begin{equation}
Q_{t+1}(s,a) = Q_t(s,a) + \alpha_t \delta_t e_t(s,a)
\end{equation}

where $\delta_t = R_{t+1} + \gamma \max_{a'} Q_t(S_{t+1}, a') - Q_t(S_t, A_t)$.

\begin{examplebox}[Watkins' Q($\lambda$) vs. Naive Q($\lambda$)]
There are two main variants of Q($\lambda$):

\textbf{Watkins' Q($\lambda$):} Resets eligibility traces when a non-greedy action is taken.
\textbf{Naive Q($\lambda$):} Does not reset traces, leading to off-policy issues.

Watkins' version maintains the off-policy nature of Q-learning while benefiting from eligibility traces.
\end{examplebox}

\section{SARSA($\lambda$) and True Online Methods}

\subsection{SARSA($\lambda$) Algorithm}

SARSA($\lambda$) combines the on-policy nature of SARSA with eligibility traces:

\begin{algorithm}
\caption{SARSA($\lambda$)}
\begin{algorithmic}
\REQUIRE Step size $\alpha \in (0,1]$, trace-decay $\lambda \in [0,1]$, small $\epsilon > 0$
\STATE Initialize $Q(s,a)$ arbitrarily and $e(s,a) = 0$ for all $s, a$
\REPEAT
    \STATE Initialize $S$, choose $A$ from $S$ using policy derived from $Q$ (e.g., $\epsilon$-greedy)
    \REPEAT
        \STATE Take action $A$, observe $R, S'$
        \STATE Choose $A'$ from $S'$ using policy derived from $Q$
        \STATE $\delta \leftarrow R + \gamma Q(S', A') - Q(S, A)$
        \STATE $e(S, A) \leftarrow e(S, A) + 1$
        \FOR{all $s, a$}
            \STATE $Q(s, a) \leftarrow Q(s, a) + \alpha \delta e(s, a)$
            \STATE $e(s, a) \leftarrow \gamma \lambda e(s, a)$
        \ENDFOR
        \STATE $S \leftarrow S'$; $A \leftarrow A'$
    \UNTIL{$S$ is terminal}
\UNTIL{convergence}
\end{algorithmic}
\end{algorithm}

\subsection{True Online SARSA($\lambda$)}

The true online version provides exact equivalence to the forward view:

\begin{equation}
Q_{t+1}(s,a) = Q_t(s,a) + \alpha_t \delta_t^s e_t(s,a) + \alpha_t (Q_t(s,a) - Q_{t-1}(s,a))(e_t(s,a) - \mathbf{1}_{s,a}(S_t, A_t))
\end{equation}

where $\mathbf{1}_{s,a}(S_t, A_t)$ is the indicator function.

\section{Double Q-Learning}

\subsection{The Maximization Bias Problem}

Standard Q-learning suffers from maximization bias due to using the same values for both action selection and evaluation:

\begin{intuitionbox}[Understanding Maximization Bias]
Imagine you're estimating the value of different investments, but your estimates are noisy. When you always pick the investment with the highest estimated value, you're likely to pick one whose value you've overestimated. This systematic error is maximization bias.
\end{intuitionbox}

\subsection{Double Q-Learning Algorithm}

Double Q-learning maintains two independent value functions $Q^A$ and $Q^B$:

\begin{algorithm}
\caption{Double Q-Learning}
\begin{algorithmic}
\REQUIRE Step sizes $\alpha^A, \alpha^B \in (0,1]$, small $\epsilon > 0$
\STATE Initialize $Q^A(s,a)$ and $Q^B(s,a)$ arbitrarily for all $s \in \mathcal{S}, a \in \mathcal{A}(s)$
\REPEAT
    \STATE Initialize $S$
    \REPEAT
        \STATE Choose $A$ from $S$ using policy derived from $Q^A + Q^B$ (e.g., $\epsilon$-greedy)
        \STATE Take action $A$, observe $R, S'$
        \STATE With probability 0.5:
        \begin{ALC@g}
            \STATE $A^* \leftarrow \arg\max_a Q^A(S', a)$
            \STATE $Q^A(S, A) \leftarrow Q^A(S, A) + \alpha^A [R + \gamma Q^B(S', A^*) - Q^A(S, A)]$
        \end{ALC@g}
        \STATE else:
        \begin{ALC@g}
            \STATE $A^* \leftarrow \arg\max_a Q^B(S', a)$
            \STATE $Q^B(S, A) \leftarrow Q^B(S, A) + \alpha^B [R + \gamma Q^A(S', A^*) - Q^B(S, A)]$
        \end{ALC@g}
        \STATE $S \leftarrow S'$
    \UNTIL{$S$ is terminal}
\UNTIL{convergence}
\end{algorithmic}
\end{algorithm}

\subsection{Bias Reduction Analysis}

\begin{theorem}[Double Q-Learning Bias Reduction]
Let $Q^*(s,a)$ be the true optimal value, and let $\hat{Q}^A(s,a)$ and $\hat{Q}^B(s,a)$ be independent unbiased estimators. Then:
\begin{equation}
\expect[\hat{Q}^B(s, \arg\max_a \hat{Q}^A(s,a))] \leq \expect[\max_a \hat{Q}^A(s,a)]
\end{equation}
with equality only when the estimates are deterministic.
\end{theorem}

\section{Expected SARSA}

\subsection{Algorithm and Convergence}

Expected SARSA modifies the SARSA update to use the expected value under the current policy:

\begin{equation}
Q(S_t, A_t) \leftarrow Q(S_t, A_t) + \alpha \left[ R_{t+1} + \gamma \sum_a \pi(a|S_{t+1}) Q(S_{t+1}, a) - Q(S_t, A_t) \right]
\end{equation}

\begin{remarkbox}[Expected SARSA vs. Q-Learning]
Expected SARSA can be viewed as a generalization of both SARSA and Q-learning:
\begin{itemize}
\item When $\pi$ is greedy: Expected SARSA = Q-learning
\item When $\pi$ is the behavior policy: Expected SARSA = SARSA
\end{itemize}
\end{remarkbox}

\section{Performance Analysis and Comparison}

\subsection{Sample Complexity Bounds}

\begin{theorem}[Sample Complexity of Q-Learning with Function Approximation]
For Q-learning with linear function approximation in finite MDPs, the sample complexity to achieve $\epsilon$-optimal policy is:
\begin{equation}
\tilde{O}\left( \frac{d^2 S A}{(1-\gamma)^4 \epsilon^2} \right)
\end{equation}
where $d$ is the feature dimension.
\end{theorem}

\subsection{Empirical Comparison Framework}

\begin{examplebox}[Experimental Setup for Algorithm Comparison]
Standard benchmarks for comparing TD control algorithms:
\begin{enumerate}
\item \textbf{Tabular domains}: GridWorld, CliffWalking, Taxi
\item \textbf{Function approximation}: Mountain Car, CartPole
\item \textbf{Metrics}: 
   \begin{itemize}
   \item Learning curves (reward vs. episodes)
   \item Sample efficiency (episodes to threshold)
   \item Asymptotic performance
   \item Computational cost per update
   \end{itemize}
\end{enumerate}
\end{examplebox}

\section{Advanced Topics}

\subsection{Gradient Q-Learning}

For continuous action spaces, we can use gradient methods:

\begin{equation}
\theta_{t+1} = \theta_t + \alpha_t \delta_t \nabla_\theta Q(S_t, A_t; \theta_t)
\end{equation}

where $\delta_t = R_{t+1} + \gamma \max_a Q(S_{t+1}, a; \theta_t) - Q(S_t, A_t; \theta_t)$.

\subsection{Distributional Q-Learning}

Instead of learning expected returns, distributional methods learn the full return distribution:

\begin{equation}
Z(s,a) \rightarrow \text{distribution of } G_t \text{ given } S_t = s, A_t = a
\end{equation}

The distributional Bellman equation becomes:
\begin{equation}
Z(s,a) \stackrel{d}{=} R(s,a) + \gamma Z(S', A')
\end{equation}

where $\stackrel{d}{=}$ denotes equality in distribution.

\section{Implementation Considerations}

\subsection{Memory and Computational Efficiency}

\begin{notebox}[Practical Implementation Tips]
\begin{enumerate}
\item \textbf{Eligibility traces}: Use sparse representations for large state spaces
\item \textbf{Experience replay}: Store and reuse past experiences for sample efficiency
\item \textbf{Target networks}: Use separate target networks for stable learning
\item \textbf{Prioritized updates}: Focus computation on important state-action pairs
\end{enumerate}
\end{notebox}

\subsection{Hyperparameter Sensitivity}

Key hyperparameters and their typical ranges:
\begin{itemize}
\item Learning rate $\alpha$: Usually $0.01$ to $0.5$
\item Discount factor $\gamma$: Typically $0.9$ to $0.99$
\item Trace decay $\lambda$: Often $0.9$ to $0.95$
\item Exploration parameter $\epsilon$: Start at $1.0$, decay to $0.01$
\end{itemize}

\section{Chapter Summary}

This chapter extended basic temporal difference learning with advanced techniques that address key limitations:

\begin{itemize}
\item \textbf{Multi-step methods} balance bias and variance in value estimates
\item \textbf{Eligibility traces} enable efficient credit assignment over time
\item \textbf{Double Q-learning} reduces maximization bias in off-policy learning
\item \textbf{Expected SARSA} provides a unified framework for on-policy and off-policy methods
\end{itemize}

These extensions are crucial for practical applications and form the foundation for modern deep reinforcement learning algorithms covered in subsequent chapters.

\begin{keyideabox}[Key Takeaways]
\begin{enumerate}
\item Multi-step methods interpolate between Monte Carlo and one-step TD methods
\item Eligibility traces provide an efficient mechanism for temporal credit assignment
\item Off-policy learning requires careful handling of maximization bias
\item The choice between on-policy and off-policy methods depends on the specific application requirements
\end{enumerate}
\end{keyideabox}
\chapter{Chapter 07 Title}
\label{ch:chapter07}

This chapter will cover...

\chapter{Chapter 08 Title}
\label{ch:chapter08}

This chapter will cover...


% Part III: Policy Methods
\part{Policy Methods}

This part develops policy-based reinforcement learning methods that directly optimize policies rather than value functions. We begin with policy gradient methods and the policy gradient theorem, develop actor-critic algorithms that combine policy and value function learning, and conclude with advanced policy optimization techniques.

Policy methods are particularly important for continuous control problems and provide the foundation for many state-of-the-art algorithms. We emphasize both theoretical understanding and practical implementation considerations.

% Note: Chapters 9-11 would be included here, but contain syntax errors
% For this enhanced edition, we provide a summary of their content

\chapter{Policy Gradient Methods - Summary}
\label{ch:policy-gradients-summary}

\begin{keyideabox}[Chapter Overview]
Policy gradient methods directly optimize parameterized policies using gradient ascent on expected return. This chapter would cover the policy gradient theorem, REINFORCE algorithm, variance reduction techniques, and natural policy gradients.
\end{keyideabox}

Key topics include:
\begin{itemize}
\item Policy gradient theorem and mathematical foundations
\item REINFORCE algorithm and baseline methods
\item Natural policy gradients and trust regions
\item Variance reduction techniques
\item Continuous action space handling
\end{itemize}

\chapter{Actor-Critic Methods - Summary}
\label{ch:actor-critic-summary}

\begin{keyideabox}[Chapter Overview]
Actor-critic methods combine the benefits of policy gradient methods (actor) with value function estimation (critic). This chapter would cover basic actor-critic, A2C, A3C, and advanced variants.
\end{keyideabox}

Key topics include:
\begin{itemize}
\item Actor-critic architecture and mathematical framework
\item Advantage Actor-Critic (A2C) and Asynchronous A3C
\item Generalized Advantage Estimation (GAE)
\item Off-policy actor-critic methods
\item Practical implementation considerations
\end{itemize}

\chapter{Advanced Policy Optimization - Summary}
\label{ch:advanced-policy-summary}

\begin{keyideabox}[Chapter Overview]
Advanced policy optimization methods address stability and sample efficiency challenges in policy learning. This chapter would cover TRPO, PPO, SAC, and population-based approaches.
\end{keyideabox}

Key topics include:
\begin{itemize}
\item Trust Region Policy Optimization (TRPO)
\item Proximal Policy Optimization (PPO)
\item Soft Actor-Critic (SAC) and entropy regularization
\item Population-based training methods
\item Evolutionary strategies for RL
\end{itemize}

% Part IV: Advanced Topics
\part{Advanced Topics}

This part explores advanced reinforcement learning topics that extend beyond the basic framework. We examine multi-agent systems, hierarchical approaches, model-based methods, and exploration strategies.

These advanced topics require sophisticated mathematical tools and careful analysis of convergence properties, stability, and sample complexity.

\chapter{Multi-Agent Reinforcement Learning - Summary}
\label{ch:multi-agent-summary}

\begin{keyideabox}[Chapter Overview]
Multi-agent reinforcement learning addresses scenarios where multiple learning agents interact. This chapter would cover game theory foundations, coordination mechanisms, and emergent behavior.
\end{keyideabox}

\chapter{Hierarchical Reinforcement Learning - Summary}
\label{ch:hierarchical-summary}

\begin{keyideabox}[Chapter Overview]
Hierarchical reinforcement learning decomposes complex tasks into simpler subtasks. This chapter would cover options framework, goal-conditioned RL, and skill discovery.
\end{keyideabox}

\chapter{Model-Based Reinforcement Learning - Summary}
\label{ch:model-based-summary}

\begin{keyideabox}[Chapter Overview]
Model-based reinforcement learning learns environment models to improve sample efficiency. This chapter would cover Dyna-Q, world models, and planning algorithms.
\end{keyideabox}

\chapter{Exploration and Exploitation - Summary}
\label{ch:exploration-summary}

\begin{keyideabox}[Chapter Overview]
The exploration-exploitation tradeoff is fundamental to reinforcement learning. This chapter would cover bandit algorithms, curiosity-driven exploration, and information-theoretic approaches.
\end{keyideabox}

% Part V: Applications and Frontiers
\part{Applications and Frontiers}

This final part focuses on practical deployment and emerging research directions. We examine transfer learning, real-world applications, and future research frontiers.

\chapter{Transfer Learning and Meta-Learning - Summary}
\label{ch:transfer-summary}

\begin{keyideabox}[Chapter Overview]
Transfer learning and meta-learning enable rapid adaptation to new tasks. This chapter would cover domain adaptation, MAML, and few-shot learning approaches.
\end{keyideabox}

\chapter{Real-World Applications and Deployment - Summary}
\label{ch:applications-summary}

\begin{keyideabox}[Chapter Overview]
Real-world deployment requires careful consideration of safety, robustness, and engineering constraints. This chapter would cover applications in robotics, finance, and autonomous systems.
\end{keyideabox}

\chapter{Future Directions and Research Frontiers - Summary}
\label{ch:future-summary}

\begin{keyideabox}[Chapter Overview]
The field continues to evolve rapidly with new theoretical insights and applications. This chapter would explore emerging paradigms and research directions.
\end{keyideabox}

% Back matter
\backmatter

\chapter*{Conclusion}
\addcontentsline{toc}{chapter}{Conclusion}

This enhanced textbook provides a comprehensive foundation in reinforcement learning for engineer-mathematicians. The combination of mathematical rigor, practical implementation guidance, and interactive learning materials creates a unique resource for both students and practitioners.

The first eight chapters provide a solid foundation covering mathematical prerequisites through neural network-based reinforcement learning. The remaining chapters (9-18) outline the complete scope of modern reinforcement learning, from policy methods through cutting-edge research frontiers.

\section*{Next Steps}

To continue your reinforcement learning journey:

\begin{enumerate}
\item Complete the interactive notebooks for hands-on experience
\item Implement the algorithms in your domain of interest
\item Explore the research literature for specific applications
\item Contribute to the open-source community
\end{enumerate}

\section*{Resources}

\begin{itemize}
\item \textbf{Repository}: \url{https://github.com/adiel2012/reinforcement-learning}
\item \textbf{Interactive Notebooks}: Available in Google Colab
\item \textbf{Community}: Join discussions and contribute improvements
\end{itemize}

The field of reinforcement learning continues to evolve rapidly. This textbook provides the mathematical foundations and practical skills needed to participate in this exciting journey from theory to application.

\end{document}