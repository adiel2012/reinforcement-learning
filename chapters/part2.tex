\part{Core Algorithms and Theory}

This part develops the fundamental learning algorithms that form the core of reinforcement learning. Moving beyond the dynamic programming methods of Part I, which assume complete knowledge of the MDP, we now consider algorithms that learn from experience through interaction with the environment.

We begin with Monte Carlo methods that estimate value functions from complete episodes. We then develop temporal difference learning, which enables learning from individual transitions. Finally, we examine Q-learning and SARSA, which learn action-value functions and form the foundation for more advanced algorithms.

Throughout this part, we emphasize mathematical rigor in convergence analysis while maintaining practical relevance for engineering applications. Each algorithm is developed with careful attention to assumptions, convergence conditions, and sample complexity bounds.

\chapter{Monte Carlo Methods}
\label{ch:monte-carlo}

Monte Carlo methods form the foundation of model-free reinforcement learning, enabling value function estimation from sample episodes without requiring knowledge of the environment dynamics. This chapter develops the mathematical theory of Monte Carlo estimation in the RL context, with emphasis on convergence analysis and variance reduction techniques.

\section{Monte Carlo Estimation Theory}

Monte Carlo methods estimate expectations by sampling. In reinforcement learning, we use sample episodes to estimate value functions without requiring the transition probabilities or reward function.

\subsection{Basic Monte Carlo Principle}

Consider estimating the expectation $\expect[X]$ of random variable $X$. The Monte Carlo estimator uses $n$ independent samples $X_1, \ldots, X_n$:

\begin{equation}
\hat{\mu}_n = \frac{1}{n} \sum_{i=1}^n X_i
\end{equation}

\begin{theorem}[Strong Law of Large Numbers]
If $\expect[|X|] < \infty$, then $\hat{\mu}_n \to \expect[X]$ almost surely as $n \to \infty$.
\end{theorem}

\begin{theorem}[Central Limit Theorem]
If $\text{Var}(X) = \sigma^2 < \infty$, then:
\begin{equation}
\sqrt{n}(\hat{\mu}_n - \expect[X]) \xrightarrow{d} \mathcal{N}(0, \sigma^2)
\end{equation}
\end{theorem}

\subsection{Application to Value Function Estimation}

For policy $\pi$, the value function is:
\begin{equation}
V^\pi(s) = \expect^\pi\left[\sum_{t=0}^\infty \gamma^t R_{t+1} \mid S_0 = s\right]
\end{equation}

Monte Carlo estimation uses sample returns $G_t = \sum_{k=0}^\infty \gamma^k R_{t+k+1}$ from episodes starting in state $s$ to estimate $V^\pi(s)$.

\section{First-Visit vs. Every-Visit Methods}

\subsection{First-Visit Monte Carlo}

\begin{algorithm}
\caption{First-Visit Monte Carlo Policy Evaluation}
\begin{algorithmic}
\REQUIRE Policy $\pi$ to evaluate
\STATE Initialize $V(s) \in \real$ arbitrarily for all $s \in \mathcal{S}$
\STATE Initialize $Returns(s) \leftarrow$ empty list for all $s \in \mathcal{S}$
\REPEAT
    \STATE Generate episode following $\pi$: $S_0, A_0, R_1, S_1, A_1, R_2, \ldots, S_{T-1}, A_{T-1}, R_T$
    \STATE $G \leftarrow 0$
    \FOR{$t = T-1, T-2, \ldots, 0$}
        \STATE $G \leftarrow \gamma G + R_{t+1}$
        \IF{$S_t$ not appear in $S_0, S_1, \ldots, S_{t-1}$}
            \STATE Append $G$ to $Returns(S_t)$
            \STATE $V(S_t) \leftarrow$ average$(Returns(S_t))$
        \ENDIF
    \ENDFOR
\UNTIL{convergence}
\end{algorithmic}
\end{algorithm}

\subsection{Every-Visit Monte Carlo}

Every-visit MC updates the value estimate every time a state is visited in an episode, not just the first time.

\begin{theorem}[Convergence of First-Visit Monte Carlo]
First-visit Monte Carlo converges to $V^\pi(s)$ as the number of first visits to state $s$ approaches infinity, assuming:
\begin{enumerate}
    \item Episodes are generated according to policy $\pi$
    \item Each state has non-zero probability of being the starting state
    \item Returns have finite variance
\end{enumerate}
\end{theorem}

\begin{proof}
Each first visit to state $s$ provides an unbiased sample of the return. By the strong law of large numbers, the sample average converges to the true expectation.
\end{proof}

\begin{theorem}[Convergence of Every-Visit Monte Carlo]
Every-visit Monte Carlo also converges to $V^\pi(s)$ under similar conditions, despite the correlation between visits within the same episode.
\end{theorem}

\section{Variance Reduction Techniques}

\subsection{Incremental Implementation}

Instead of storing all returns, we can update estimates incrementally:

\begin{equation}
V_{n+1}(s) = V_n(s) + \frac{1}{n+1}[G_n - V_n(s)]
\end{equation}

More generally, with step size $\alpha$:
\begin{equation}
V(s) \leftarrow V(s) + \alpha[G - V(s)]
\end{equation}

\subsection{Baseline Subtraction}

To reduce variance, we can subtract a baseline $b(s)$ that doesn't depend on the action:

\begin{equation}
G_t - b(S_t)
\end{equation}

The optimal baseline that minimizes variance is:
\begin{equation}
b^*(s) = \frac{\expect[G_t^2 \mid S_t = s]}{\expect[G_t \mid S_t = s]} = \expect[G_t \mid S_t = s] = V^\pi(s)
\end{equation}

\subsection{Control Variates}

For correlated random variable $Y$ with known expectation $\expect[Y] = \mu_Y$:
\begin{equation}
\hat{\mu}_{CV} = \hat{\mu}_X - c(\hat{\mu}_Y - \mu_Y)
\end{equation}

The optimal coefficient is:
\begin{equation}
c^* = \frac{\text{Cov}(X,Y)}{\text{Var}(Y)}
\end{equation}

\section{Importance Sampling in RL}

Importance sampling enables off-policy learning by weighting samples according to the ratio of target to behavior policy probabilities.

\subsection{Ordinary Importance Sampling}

To estimate $\expect_\pi[X]$ using samples from policy $\mu$:
\begin{equation}
\hat{\mu}_{IS} = \frac{1}{n} \sum_{i=1}^n \rho_i X_i
\end{equation}

where $\rho_i = \frac{\pi(A_i|S_i)}{\mu(A_i|S_i)}$ is the importance sampling ratio.

\begin{theorem}[Unbiasedness of Importance Sampling]
$\expect[\hat{\mu}_{IS}] = \expect_\pi[X]$ if $\mu(a|s) > 0$ whenever $\pi(a|s) > 0$.
\end{theorem}

\subsection{Weighted Importance Sampling}

To reduce variance when some importance weights are very large:
\begin{equation}
\hat{\mu}_{WIS} = \frac{\sum_{i=1}^n \rho_i X_i}{\sum_{i=1}^n \rho_i}
\end{equation}

\begin{theorem}[Bias-Variance Tradeoff]
Weighted importance sampling is biased but often has lower variance than ordinary importance sampling:
\begin{align}
\text{Bias}[\hat{\mu}_{WIS}] &\neq 0 \text{ (in general)} \\
\text{Var}[\hat{\mu}_{WIS}] &\leq \text{Var}[\hat{\mu}_{IS}] \text{ (typically)}
\end{align}
\end{theorem}

\subsection{Per-Decision Importance Sampling}

For episodic tasks, the importance sampling ratio for a complete episode is:
\begin{equation}
\rho_{t:T-1} = \prod_{k=t}^{T-1} \frac{\pi(A_k|S_k)}{\mu(A_k|S_k)}
\end{equation}

This can have very high variance. Per-decision importance sampling uses only the relevant portion of the trajectory.

\section{Off-Policy Monte Carlo Methods}

\subsection{Off-Policy Policy Evaluation}

\begin{algorithm}
\caption{Off-Policy Monte Carlo Policy Evaluation}
\begin{algorithmic}
\REQUIRE Target policy $\pi$, behavior policy $\mu$
\STATE Initialize $V(s) \in \real$ arbitrarily for all $s \in \mathcal{S}$
\STATE Initialize $C(s) \leftarrow 0$ for all $s \in \mathcal{S}$
\REPEAT
    \STATE Generate episode using $\mu$: $S_0, A_0, R_1, \ldots, S_{T-1}, A_{T-1}, R_T$
    \STATE $G \leftarrow 0$
    \STATE $W \leftarrow 1$
    \FOR{$t = T-1, T-2, \ldots, 0$}
        \STATE $G \leftarrow \gamma G + R_{t+1}$
        \STATE $C(S_t) \leftarrow C(S_t) + W$
        \STATE $V(S_t) \leftarrow V(S_t) + \frac{W}{C(S_t)}[G - V(S_t)]$
        \STATE $W \leftarrow W \frac{\pi(A_t|S_t)}{\mu(A_t|S_t)}$
        \IF{$W = 0$}
            \STATE break
        \ENDIF
    \ENDFOR
\UNTIL{convergence}
\end{algorithmic}
\end{algorithm}

\subsection{Off-Policy Monte Carlo Control}

\begin{algorithm}
\caption{Off-Policy Monte Carlo Control}
\begin{algorithmic}
\STATE Initialize $Q(s,a) \in \real$ arbitrarily for all $s,a$
\STATE Initialize $C(s,a) \leftarrow 0$ for all $s,a$
\STATE Initialize $\pi(s) \leftarrow \argmax_a Q(s,a)$ for all $s$
\REPEAT
    \STATE Choose any soft policy $\mu$ (e.g., $\epsilon$-greedy)
    \STATE Generate episode using $\mu$
    \STATE $G \leftarrow 0$
    \STATE $W \leftarrow 1$
    \FOR{$t = T-1, T-2, \ldots, 0$}
        \STATE $G \leftarrow \gamma G + R_{t+1}$
        \STATE $C(S_t, A_t) \leftarrow C(S_t, A_t) + W$
        \STATE $Q(S_t, A_t) \leftarrow Q(S_t, A_t) + \frac{W}{C(S_t, A_t)}[G - Q(S_t, A_t)]$
        \STATE $\pi(S_t) \leftarrow \argmax_a Q(S_t, a)$
        \IF{$A_t \neq \pi(S_t)$}
            \STATE break
        \ENDIF
        \STATE $W \leftarrow W \frac{1}{\mu(A_t|S_t)}$
    \ENDFOR
\UNTIL{convergence}
\end{algorithmic}
\end{algorithm}

\section{Convergence Analysis and Sample Complexity}

\subsection{Finite Sample Analysis}

\begin{theorem}[Finite Sample Bound for Monte Carlo]
Let $V_n(s)$ be the Monte Carlo estimate after $n$ visits to state $s$. Under bounded rewards $|R| \leq R_{max}$:
\begin{equation}
\prob\left(|V_n(s) - V^\pi(s)| \geq \epsilon\right) \leq 2\exp\left(-\frac{2n\epsilon^2(1-\gamma)^2}{R_{max}^2}\right)
\end{equation}
\end{theorem}

\subsection{Asymptotic Convergence Rate}

\begin{theorem}[Central Limit Theorem for Monte Carlo]
If $\text{Var}^\pi[G_t | S_t = s] = \sigma^2(s) < \infty$, then:
\begin{equation}
\sqrt{n}(V_n(s) - V^\pi(s)) \xrightarrow{d} \mathcal{N}(0, \sigma^2(s))
\end{equation}
\end{theorem}

This gives the convergence rate $O(n^{-1/2})$, which is slower than the $O(n^{-1})$ rate achievable by temporal difference methods under certain conditions.

\subsection{Sample Complexity}

\begin{theorem}[Sample Complexity of Monte Carlo]
To achieve $\epsilon$-accurate value function estimation with probability $1-\delta$:
\begin{equation}
n \geq \frac{R_{max}^2 \log(2/\delta)}{2\epsilon^2(1-\gamma)^2}
\end{equation}
samples are sufficient.
\end{theorem}

\section{Practical Considerations}

\subsection{Exploration vs. Exploitation}

Monte Carlo control methods face the exploration-exploitation dilemma. Common approaches:

\textbf{Exploring Starts:} Assume episodes start in randomly selected state-action pairs.

\textbf{$\epsilon$-Greedy Policies:} Use soft policies that maintain exploration:
\begin{equation}
\pi(a|s) = \begin{cases}
1 - \epsilon + \frac{\epsilon}{|\mathcal{A}(s)|} & \text{if } a = \argmax_a Q(s,a) \\
\frac{\epsilon}{|\mathcal{A}(s)|} & \text{otherwise}
\end{cases}
\end{equation}

\subsection{Function Approximation}

For large state spaces, we approximate value functions:
\begin{equation}
V(s) \approx \hat{V}(s, \mathbf{w}) = \mathbf{w}^T \boldsymbol{\phi}(s)
\end{equation}

The Monte Carlo update becomes:
\begin{equation}
\mathbf{w} \leftarrow \mathbf{w} + \alpha[G_t - \hat{V}(S_t, \mathbf{w})]\nabla_\mathbf{w} \hat{V}(S_t, \mathbf{w})
\end{equation}

\begin{theorem}[Convergence with Linear Function Approximation]
Under linear function approximation with linearly independent features, Monte Carlo methods converge to the best linear approximation in the $L^2$ norm weighted by the stationary distribution.
\end{theorem}

\section{Chapter Summary}

This chapter established the foundations of Monte Carlo methods in reinforcement learning:

\begin{itemize}
    \item Monte Carlo estimation theory and convergence properties
    \item First-visit vs. every-visit methods with convergence guarantees
    \item Variance reduction techniques: baselines, control variates, importance sampling
    \item Off-policy learning through importance sampling with bias-variance analysis
    \item Sample complexity bounds and convergence rates
    \item Practical considerations for exploration and function approximation
\end{itemize}

Monte Carlo methods provide unbiased estimates and are conceptually simple, but they require complete episodes and have slower convergence than temporal difference methods. The next chapter develops temporal difference learning, which enables learning from individual transitions.
\chapter{Temporal Difference Learning}
\label{ch:temporal-difference}

Temporal Difference (TD) learning combines ideas from Monte Carlo methods and dynamic programming, enabling learning from incomplete episodes while maintaining the model-free nature of Monte Carlo methods. This chapter develops the mathematical theory of TD learning with emphasis on convergence analysis and the fundamental bias-variance tradeoff.

\section{TD(0) Algorithm and Mathematical Analysis}

\subsection{Basic TD(0) Update}

The core insight of temporal difference learning is to use the current estimate of the successor state's value to update the current state's value:

\begin{equation}
V(S_t) \leftarrow V(S_t) + \alpha [R_{t+1} + \gamma V(S_{t+1}) - V(S_t)]
\end{equation}

The TD error is defined as:
\begin{equation}
\delta_t = R_{t+1} + \gamma V(S_{t+1}) - V(S_t)
\end{equation}

\begin{algorithm}
\caption{Tabular TD(0) Policy Evaluation}
\begin{algorithmic}
\REQUIRE Policy $\pi$ to evaluate, step size $\alpha \in (0,1]$
\STATE Initialize $V(s) \in \real$ arbitrarily for all $s \in \mathcal{S}$, except $V(\text{terminal}) = 0$
\REPEAT
    \STATE Initialize $S$
    \REPEAT
        \STATE $A \leftarrow$ action given by $\pi$ for $S$
        \STATE Take action $A$, observe $R, S'$
        \STATE $V(S) \leftarrow V(S) + \alpha[R + \gamma V(S') - V(S)]$
        \STATE $S \leftarrow S'$
    \UNTIL{$S$ is terminal}
\UNTIL{convergence or sufficient accuracy}
\end{algorithmic}
\end{algorithm}

\subsection{Relationship to Bellman Equation}

The TD(0) update can be viewed as a stochastic approximation to the Bellman equation. The expected TD update is:

\begin{align}
\expect[\delta_t | S_t = s] &= \expect[R_{t+1} + \gamma V(S_{t+1}) - V(S_t) | S_t = s] \\
&= \sum_{s',r} p(s',r|s,\pi(s))[r + \gamma V(s') - V(s)] \\
&= (T^\pi V)(s) - V(s)
\end{align}

where $T^\pi$ is the Bellman operator for policy $\pi$.

\subsection{Convergence Analysis}

\begin{theorem}[Convergence of TD(0) - Tabular Case]
For the tabular case with appropriate step size sequence $\{\alpha_t\}$ satisfying:
\begin{align}
\sum_{t=0}^\infty \alpha_t &= \infty \\
\sum_{t=0}^\infty \alpha_t^2 &< \infty
\end{align}
TD(0) converges to $V^\pi$ with probability 1.
\end{theorem}

\begin{proof}[Proof Sketch]
The proof uses stochastic approximation theory. Define the ODE:
\begin{equation}
\frac{dV}{dt} = \expect[\delta_t | V] = T^\pi V - V
\end{equation}
Since $T^\pi$ is a contraction, the unique fixed point is $V^\pi$. The stochastic approximation theorem ensures convergence of the discrete updates to the ODE solution.
\end{proof}

\section{Bias-Variance Tradeoff in TD Methods}

\subsection{Bias Analysis}

TD(0) uses the biased estimate $R_{t+1} + \gamma V(S_{t+1})$ as a target for $V(S_t)$, while Monte Carlo uses the unbiased estimate $G_t$.

\begin{theorem}[Bias of TD Target]
The TD target $R_{t+1} + \gamma V(S_{t+1})$ has bias:
\begin{equation}
\text{Bias}[R_{t+1} + \gamma V(S_{t+1})] = \gamma[\hat{V}(S_{t+1}) - V^\pi(S_{t+1})]
\end{equation}
where $\hat{V}$ is the current estimate.
\end{theorem}

\subsection{Variance Analysis}

\begin{theorem}[Variance Comparison]
Under the assumption that value function errors are small, the variance of the TD target is approximately:
\begin{equation}
\text{Var}[R_{t+1} + \gamma V(S_{t+1})] \approx \text{Var}[R_{t+1}]
\end{equation}
while the Monte Carlo target has variance:
\begin{equation}
\text{Var}[G_t] = \text{Var}\left[\sum_{k=0}^\infty \gamma^k R_{t+k+1}\right]
\end{equation}
which is typically much larger.
\end{theorem}

\subsection{Mean Squared Error Decomposition}

\begin{equation}
\text{MSE} = \text{Bias}^2 + \text{Variance} + \text{Noise}
\end{equation}

TD methods trade increased bias for reduced variance, often resulting in lower overall MSE and faster convergence.

\section{TD(λ) and Eligibility Traces}

TD(λ) provides a family of algorithms that interpolate between TD(0) and Monte Carlo methods through the use of eligibility traces.

\subsection{Forward View: n-step Returns}

The n-step return combines rewards from the next n steps with the estimated value of the state reached after n steps:

\begin{equation}
G_t^{(n)} = R_{t+1} + \gamma R_{t+2} + \cdots + \gamma^{n-1} R_{t+n} + \gamma^n V(S_{t+n})
\end{equation}

The n-step TD update is:
\begin{equation}
V(S_t) \leftarrow V(S_t) + \alpha[G_t^{(n)} - V(S_t)]
\end{equation}

\subsection{λ-Return}

The λ-return combines all n-step returns:
\begin{equation}
G_t^\lambda = (1-\lambda) \sum_{n=1}^\infty \lambda^{n-1} G_t^{(n)}
\end{equation}

\begin{theorem}[λ-Return Properties]
The λ-return satisfies:
\begin{align}
G_t^\lambda &= R_{t+1} + \gamma[(1-\lambda)V(S_{t+1}) + \lambda G_{t+1}^\lambda] \\
\lim_{\lambda \to 0} G_t^\lambda &= R_{t+1} + \gamma V(S_{t+1}) \quad \text{(TD(0))} \\
\lim_{\lambda \to 1} G_t^\lambda &= G_t \quad \text{(Monte Carlo)}
\end{align}
\end{theorem}

\subsection{Backward View: Eligibility Traces}

Eligibility traces provide an online, incremental implementation of TD(λ):

\begin{align}
\delta_t &= R_{t+1} + \gamma V(S_{t+1}) - V(S_t) \\
e_t(s) &= \begin{cases}
\gamma \lambda e_{t-1}(s) + 1 & \text{if } s = S_t \\
\gamma \lambda e_{t-1}(s) & \text{if } s \neq S_t
\end{cases} \\
V(s) &\leftarrow V(s) + \alpha \delta_t e_t(s) \quad \forall s
\end{align}

\begin{algorithm}
\caption{TD(λ) with Eligibility Traces}
\begin{algorithmic}
\REQUIRE Policy $\pi$, step size $\alpha$, trace decay $\lambda$
\STATE Initialize $V(s) \in \real$ arbitrarily for all $s$
\REPEAT
    \STATE Initialize $S$, $e(s) = 0$ for all $s$
    \REPEAT
        \STATE $A \leftarrow$ action given by $\pi$ for $S$
        \STATE Take action $A$, observe $R, S'$
        \STATE $\delta \leftarrow R + \gamma V(S') - V(S)$
        \STATE $e(S) \leftarrow e(S) + 1$
        \FOR{all $s$}
            \STATE $V(s) \leftarrow V(s) + \alpha \delta e(s)$
            \STATE $e(s) \leftarrow \gamma \lambda e(s)$
        \ENDFOR
        \STATE $S \leftarrow S'$
    \UNTIL{$S$ is terminal}
\UNTIL{convergence}
\end{algorithmic}
\end{algorithm}

\subsection{Equivalence Theorem}

\begin{theorem}[Forward-Backward Equivalence]
Under certain conditions, the forward view (using λ-returns) and backward view (using eligibility traces) produce identical updates when applied offline to a complete episode.
\end{theorem}

\section{Convergence Theory for Linear Function Approximation}

When the state space is large, we use function approximation:
\begin{equation}
V(s) \approx \hat{V}(s, \mathbf{w}) = \mathbf{w}^T \boldsymbol{\phi}(s)
\end{equation}

The TD(0) update becomes:
\begin{equation}
\mathbf{w}_{t+1} = \mathbf{w}_t + \alpha[R_{t+1} + \gamma \mathbf{w}_t^T \boldsymbol{\phi}(S_{t+1}) - \mathbf{w}_t^T \boldsymbol{\phi}(S_t)]\boldsymbol{\phi}(S_t)
\end{equation}

\subsection{Projected Bellman Equation}

Under linear function approximation, TD(0) converges to the solution of the projected Bellman equation:
\begin{equation}
\mathbf{w}^* = \arg\min_\mathbf{w} \|\boldsymbol{\Phi}\mathbf{w} - T^\pi(\boldsymbol{\Phi}\mathbf{w})\|_{\mathbf{D}}^2
\end{equation}

where $\boldsymbol{\Phi}$ is the feature matrix and $\mathbf{D}$ is a diagonal matrix of state visitation probabilities.

\begin{theorem}[Convergence of Linear TD(0)]
Under linear function approximation, TD(0) converges to:
\begin{equation}
\mathbf{w}^* = (\boldsymbol{\Phi}^T \mathbf{D} \boldsymbol{\Phi})^{-1} \boldsymbol{\Phi}^T \mathbf{D} \mathbf{r}^\pi
\end{equation}
where $\mathbf{r}^\pi$ is the expected reward vector.
\end{theorem}

\subsection{Error Bounds}

\begin{theorem}[Approximation Error Bound]
Let $V^*$ be the optimal value function and $\hat{V}^*$ be the best linear approximation. Then:
\begin{equation}
\|V^\pi - \hat{V}^\pi\|_{\mathbf{D}} \leq \frac{1}{1-\gamma} \min_\mathbf{w} \|V^\pi - \boldsymbol{\Phi}\mathbf{w}\|_{\mathbf{D}}
\end{equation}
\end{theorem}

\section{Comparison with Monte Carlo and DP Methods}

\subsection{Computational Complexity}

\begin{center}
\begin{tabular}{lccc}
\toprule
Method & Memory & Computation per Step & Episode Completion \\
\midrule
DP & $O(|\mathcal{S}|^2|\mathcal{A}|)$ & $O(|\mathcal{S}|^2|\mathcal{A}|)$ & Not Required \\
MC & $O(|\mathcal{S}|)$ & $O(1)$ & Required \\
TD & $O(|\mathcal{S}|)$ & $O(1)$ & Not Required \\
\bottomrule
\end{tabular}
\end{center}

\subsection{Sample Efficiency}

\begin{theorem}[Sample Complexity Comparison]
Under certain regularity conditions:
\begin{itemize}
    \item TD methods: $O(\frac{1}{\epsilon^2(1-\gamma)^2})$ samples for $\epsilon$-accuracy
    \item MC methods: $O(\frac{1}{\epsilon^2(1-\gamma)^4})$ samples for $\epsilon$-accuracy
\end{itemize}
\end{theorem}

TD methods often have better sample efficiency due to lower variance, despite being biased.

\subsection{Bootstrapping vs. Sampling}

\textbf{Bootstrapping:} Using estimates of successor states (DP, TD)
\textbf{Sampling:} Using actual experience (MC, TD)

TD methods combine both, leading to:
\begin{itemize}
    \item Faster learning than MC (bootstrapping)
    \item Model-free nature (sampling)
    \item Online learning capability
\end{itemize}

\section{Advanced Topics}

\subsection{Multi-step Methods}

The n-step TD methods generalize between TD(0) and Monte Carlo:
\begin{equation}
V(S_t) \leftarrow V(S_t) + \alpha[G_t^{(n)} - V(S_t)]
\end{equation}

\begin{theorem}[Optimal Step Size]
For n-step methods, there exists an optimal n that minimizes mean squared error, typically $n \in [3, 10]$ for many problems.
\end{theorem}

\subsection{True Online TD(λ)}

The classical TD(λ) is not equivalent to the forward view when using function approximation. True online TD(λ) corrects this:

\begin{align}
\mathbf{w}_{t+1} &= \mathbf{w}_t + \alpha \delta_t \mathbf{z}_t + \alpha(\mathbf{w}_t^T \boldsymbol{\phi}_t - \mathbf{w}_{t-1}^T \boldsymbol{\phi}_t)(\mathbf{z}_t - \boldsymbol{\phi}_t) \\
\mathbf{z}_{t+1} &= \gamma \lambda \mathbf{z}_t + \boldsymbol{\phi}_{t+1} - \alpha \gamma \lambda (\mathbf{z}_t^T \boldsymbol{\phi}_{t+1})\boldsymbol{\phi}_{t+1}
\end{align}

\subsection{Gradient TD Methods}

To handle function approximation more rigorously, gradient TD methods minimize the mean squared projected Bellman error:

\begin{align}
\text{MSPBE}(\mathbf{w}) &= \|\boldsymbol{\Pi}(\mathbf{T}^\pi \hat{\mathbf{v}} - \hat{\mathbf{v}})\|_{\mathbf{D}}^2 \\
\nabla \text{MSPBE}(\mathbf{w}) &= 2\boldsymbol{\Phi}^T \mathbf{D} (\boldsymbol{\Pi}(\mathbf{T}^\pi \hat{\mathbf{v}} - \hat{\mathbf{v}}))
\end{align}

\section{Chapter Summary}

This chapter developed the mathematical foundations of temporal difference learning:

\begin{itemize}
    \item TD(0) algorithm with convergence analysis using stochastic approximation theory
    \item Bias-variance tradeoff analysis showing TD's advantage in variance reduction
    \item TD(λ) and eligibility traces providing a spectrum between TD(0) and Monte Carlo
    \item Convergence theory for linear function approximation with error bounds
    \item Comparative analysis with Monte Carlo and dynamic programming methods
    \item Advanced topics including multi-step methods and gradient TD approaches
\end{itemize}

Temporal difference learning provides the foundation for many modern RL algorithms, combining the best aspects of Monte Carlo and dynamic programming approaches. The next chapter extends these ideas to action-value methods with Q-learning and SARSA.
\chapter{Chapter 06 Title}
\label{ch:chapter06}

This chapter will cover...
